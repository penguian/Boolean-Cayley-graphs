\documentclass[12pt,a4paper]{article}
%
\usepackage{amsmath}
\usepackage{amssymb}
\usepackage{amsthm}
\usepackage{graphicx}
%
\usepackage{hyperref}
%\usepackage{url}
%
\newcommand{\mb}[1]{\mathbb{#1}}
\newcommand{\mf}[1]{\mathbf{#1}}
\newcommand{\oE}{\mf{\operatorname{E}}}
\newcommand{\G}{\mb{G}}
\newcommand{\R}{\mb{R}}
\newcommand{\Z}{\mb{Z}}
\newcommand{\Rep}{P}
\newcommand{\V}[1]{\underline{#1}}
\newcommand{\abs}[1]{\left| #1 \right|}
\newcommand{\To}{\rightarrow}
%
\newtheorem*{lemma}{Lemma}
\newtheorem{Lemma}{Lemma}
\newtheorem*{theorem}{Theorem}
\newtheorem{Theorem}[Lemma]{Theorem}
\newtheorem*{conjecture}{Conjecture}
\newtheorem{Conjecture}{Conjecture}
\newtheorem*{corollary}{Corollary}
\newtheorem{Corollary}[Lemma]{Corollary}
\newtheorem*{remark}{Remark}
\newtheorem{Remark}[Lemma]{Remark}
\newtheorem*{definition}{Definition}
\newtheorem{Definition}{Definition}
\newtheorem*{propertyonea}{Property 1a}
\newtheorem{Property}{Property}
\newtheorem{Question}{Question}
\newtheorem{Questions}[Question]{Questions}
%
\newenvironment{proofof}[1]{\noindent\emph{Proof of #1.}}{\qed}

\title{Twin bent functions, strongly regular Cayley graphs, and Hurwitz-Radon theory}

\author{
Paul~C.~Leopardi
\thanks{School of Mathematical and Physical Sciences, The University of Newcastle.
\protect\url{mailto:paul.leopardi@newcastle.edu.au}}
%\protect\texttt{mailto:paul.\-leopardi@anu.edu.au}}
}

\date{Submitted: TBD}

\begin{document}

\maketitle

\begin{abstract}
%
The real monomial representations of Clifford algebras
give rise to two sequences of bent functions.
For each of these sequences, the corresponding Cayley graphs are 
strongly regular graphs, and the corresponding sequences of strongly regular graph parameters
$(\nu,k,\lambda,\mu)$ coincide.
Even so, the corresponding graphs in the two sequences are not isomorphic, except in the first 3 cases.
The proof of this non-isomorphism is a simple consequence of a theorem of Hurwitz and Radon.  
%
\end{abstract}

\section{Introduction}
\label{sec-intro}

A recent paper of the author \cite{Leo14Constructions} describes a generalization of
Williamson's construction for Hada\-mard matrices \cite{Wil44}
using the real monomial representation of the basis elements of the Clifford algebras $\R_{m,m}$.

In that paper, the following conjectures appear:

\begin{Conjecture}\label{conjecture-1}
%
For all $m \geqslant 0$ there is a permutation $\pi$ of the set of $4^m$ canonical basis matrices,
that sends an amicable pair of basis matrices with disjoint support to an anti-amicable pair, and vice-versa.
%
\end{Conjecture}

\begin{Conjecture}\label{conjecture-2}
%
For all $m \geqslant 0,$ 
for the Clifford algebra $\R_{m,m},$ the subset of transversal graphs that are 
not self-edge-colour-complementary
can be arranged into a set of pairs of graphs with each member of the pair 
being edge-colour-complementary to the other member.
%
\end{Conjecture}

\begin{Conjecture}\label{conjecture-3}
%
For all $m \geqslant 0,$ 
for the Clifford algebra $\R_{m,m},$ if a graph $T$ exists amongst the transversal graphs,
then so does at least one graph with edge colours complementary to those of $T$.
%
\end{Conjecture}
A transversal graph is a subgraph of $\varDelta_m$ which is a complete graph of order $2^m$.

TBD: definitions

In this paper it is proved that Conjecture~\ref{conjecture-3} is false, and as a consequence, so are Conjectures~\ref{conjecture-1}
and~\ref{conjecture-2}.

\begin{Question}
\label{Question-1}
Consider the sequence of edge-coloured graphs $\varDelta_m$ $(m \geqslant 1)$ as defined in \cite{Leo14Constructions},
each with red subgraph $\varDelta_m[-1],$ and blue subgraph $\varDelta_m[1].$
For which $m \geqslant 1$ is there an automorphism of $\varDelta_m$ that swaps the subgraphs $\varDelta_m[-1]$ and $\varDelta_m[1]$?
\end{Question}
 


The paper is organized as follows.

%\newpage
\section{Signed groups and Clifford algebras}
\label{sec-A}

\section{A signed group and its real monomial representation}
\label{sec-first-bent}
%
The following definitions and results appear in the paper on Hada\-mard matrices and \index{Clifford algebras} \cite{Leo14Constructions},
and are presented here for completeness, since they are used below. 
Further details and proofs can be found in that paper, and in the paper on bent functions and Clifford algebras,
unless otherwise noted.

The signed group
$\G_{p,q}$ of order $2^{1+p+q}$ 
is extension of $\Z_2$ by $\Z_2^{p+q}$,
defined by the signed group presentation
%
\begin{align*}
\G_{p,q} := \bigg\langle \ 
&\mf{e}_{\{k\}}\ (k \in S_{p,q})\ \mid
\\
&\mf{e}_{\{k\}}^2 = -1\ (k < 0), \quad \mf{e}_{\{k\}}^2 = 1\ (k > 0),
\\
&\mf{e}_{\{j\}}\mf{e}_{\{k\}} = -\mf{e}_{\{k\}}\mf{e}_{\{j\}}\ (j \neq k) \bigg\rangle,
\end{align*}
%
where $S_{p,q} := \{-q,\ldots,-1,1,\ldots,p\}.$

The following construction of the real monomial representation $\Rep(\G_{m,m})$
of the group $\G_{m,m}$ is used in \cite{Leo14Constructions}.

The $2 \times 2$ orthogonal matrices
\begin{align*}
\oE_1 :=
\left[
\begin{array}{cc}
. & - \\
1 & .
\end{array}
\right],
\quad
\oE_2 :=
\left[
\begin{array}{cc}
. & 1 \\
1 & .
\end{array}
\right]
\end{align*}
generate $\Rep(\G_{1,1}),$ the real monomial representation of group $\G_{1,1}.$
The cosets of $\{\pm I\} \equiv \Z_2$ in $\Rep(\G_{1,1})$ are
ordered using a pair of bits, as follows.
\begin{align*}
0 &\leftrightarrow 00 \leftrightarrow \{ \pm I \},
\\
1 &\leftrightarrow 01 \leftrightarrow \{ \pm \oE_1 \},
\\
2 &\leftrightarrow 10 \leftrightarrow \{ \pm \oE_2 \},
\\
3 &\leftrightarrow 11 \leftrightarrow \{ \pm \oE_1 \oE_2 \}.
\end{align*}

For $m > 1$,
the real monomial representation $\Rep(\G_{m,m})$ of the 
group $\G_{m,m}$ consists of matrices of the form $G_1 \otimes G_{m-1}$
with $G_1$ in $\Rep(\G_{1,1})$ and $G_{m-1}$ in $\Rep(\G_{m-1,m-1}).$
The cosets of $\{\pm I\} \equiv \Z_2$ in $\Rep(\G_{m,m})$ are
ordered by concatenation of pairs of bits, 
where each pair of bits uses the ordering as per $\Rep(\G_{1,1}),$
and the pairs are ordered as follows.
\begin{align*}
0 &\leftrightarrow 00 \ldots 00 \leftrightarrow \{ \pm I \},
\\
1 &\leftrightarrow 00 \ldots 01 \leftrightarrow \{ \pm I_{(2)}^{\otimes {(m-1)}} \otimes  \oE_1 \},
\\
2 &\leftrightarrow 00 \ldots 10 \leftrightarrow \{ \pm I_{(2)}^{\otimes {(m-1)}} \otimes  \oE_2 \},
\\
&\ldots
\\
2^{2m} - 1 &\leftrightarrow 11 \ldots 11 \leftrightarrow \{ \pm (\oE_1 \oE_2)^{\otimes {m}} \}.
\end{align*}
This ordering is called 
the \emph{Kronecker product ordering} of the cosets of $\{\pm I\}$ in $\Rep(\G_{m,m}).$

Recall that
the group $\G_{m,m}$ and its real monomial representation $\Rep(\G_{m,m})$ 
satisfy the following properties.
\begin{enumerate}
\item 
Pairs of elements of $\G_{m,m}$ (and therefore $\Rep(\G_{m,m})$) either commute or anti\-commute:
for $g, h \in \G_{m,m},$ either $h g = g h$ or $h g = - g h.$
\item
The matrices $E \in \Rep(\G_{m,m})$ are orthogonal: $E E^T = E^T E = I.$
\item
The matrices $E \in \Rep(\G_{m,m})$ are either symmetric and square to give $I$ or 
skew and square to give $-I$: either $E^T = E$ and $E^2 =I$ or $E^T = -E$ and $E^2 = -I.$
\end{enumerate}

Taking the positive signed element of each of the $2^{2m}$ cosets listed above
defines a transversal of $\{\pm I\}$ in $\Rep(\G_{m,m})$
which is also a monomial basis for the real representation of the Clifford algebra $\R_{m,m}$ in 
Kronecker product order,
called this ordered monomial basis the \emph{positive signed basis} of $\Rep(\R_{m,m}).$ 

\begin{definition}\label{definition-gamma}
We define the function $\gamma_m : \Z_{2^{2 m}} \To \Rep(\G_{m,m})$ 
to choose the corresponding basis matrix from the positive signed basis of $\Rep(\R_{m,m}),$
using the Kronecker product ordering.
This ordering also defines a corresponding function on $\Z_2^{2 m},$
which we also call $\gamma_m.$
\end{definition}

\section{Two bent functions}
\label{sec-second-bent}

We now define two functions, $\sigma_m$ and $\tau_m$ on $\Z_2^{2 m},$
and show that both of these are bent.

TBD Bent function: \cite[p. 74]{Dil74}.

The first function, $\sigma_m$ is defined and shown to be bent in \cite{Leo14Constructions}.
We repeat the definition here.

\begin{definition}\label{definition-sign-of-square-function}
We use the basis element selection function $\gamma_m$ of Definition~\ref{definition-gamma} to 
define the \emph{sign-of-square} function $\sigma_m : \Z_2^{2 m} \To \Z_2$ as
\begin{align*}
\sigma_m(i) &:=
\begin{cases}
1 \leftrightarrow \gamma_m(i)^2 = -I
\\
0 \leftrightarrow \gamma_m(i)^2 = I,
\end{cases}
\end{align*}
for all $i$ in $\Z_2^{2 m}$.
\end{definition}
In the paper on bent functions \cite{Leo15Bent}, the following two properties are noted.
\begin{enumerate}
 \item 
Since each $\gamma_m(i)$ is orthogonal,
$\sigma_m(i) = 1$ if and only if $\gamma_m(i)$ is skew.
 \item 
For $i \in \Z_{2^{2m}},$ $\sigma_m(i) = 1$ if and only if the number of
1 digits in  the base 4 representation of $i$ is odd.
\end{enumerate}

The basis element selection function $\gamma_m$ also gives rise to a second function,
$\tau_m$ on $\Z_{2^{2 m}}.$

The second function, $\tau_m$ is defined and shown to be bent in the paper on bent functions \cite{Leo15Bent}.
We repeat the definition here.

\begin{definition}\label{definition-non-diagonal-symmetry-function}
We define the \emph{non-diagonal-symmetry} function $\tau_m$ on $\Z_{2^{2 m}}$ and $\Z_2^{2 m}$
as follows.

For $i$ in $\Z_2^2$:
\begin{align*}
&\tau_1(i) :=
\begin{cases}
1 &\text{if~}i = 10,\text{~so that~}\gamma_1(i) = \pm \oE_2,
\\
0 &\text{otherwise}.
\end{cases}
\end{align*}

For $i$ in $\Z_2^{2 m - 2}$:
\begin{align*}
&\tau_m (00 \odot i) := \tau_{m-1}(i), 
\\
&\tau_m (01 \odot i) := \sigma_{m-1}(i),
\\
&\tau_m (10 \odot i) := \sigma_{m-1}(i) + 1,
\\ 
&\tau_m (11 \odot i) := \tau_{m-1}(i).
\end{align*}
where $\odot$ denotes concatenation of bit vectors, and $\sigma$ is the sign-of-square function, as above.
\end{definition}

It is easy to verify that
$\tau_m(i) = 1$ if and only if $\gamma_m(i)$ is symmetric but not diagonal.

\section{Bent functions and strongly regular graphs}
\label{sec-strongly-regular}
%
This section examines the relationship between the bent functions $\sigma_m$ and $\tau_m$ and
the subgraphs $\varDelta_m[-1]$ and $\varDelta_m[1]$ from Question~\ref{Question-1}.

We use this result to examine the graph $\varDelta_m.$
The following two definitions appear in the previous paper \cite{Leo14Constructions}
and are repeated here for completeness.
\begin{definition}\label{definition-delta}
Let $\varDelta_m$ be the graph whose vertices are the $n^2=4^m$ 
canonical basis matrices of the real representation
of the Clifford algebra $\R_{m,m}$,
with each edge having one of two colours, $-1$ (red) and $1$ (blue):
\begin{itemize}
\item 
Matrices $A_j$ and $A_k$ are connected by a red edge if they have disjoint support and are anti-amicable,
i.e. $A_j A_k^{-1}$ is skew.
\item 
Matrices $A_j$ and $A_k$ are connected by a blue edge if they have disjoint support and are amicable,
i.e. $A_j A_k^{-1}$ is symmetric.
\item 
Otherwise there is no edge between $A_j$ and $A_k$.
\end{itemize}
We call this graph the \emph{restricted amicability / anti-amicability graph}
\index{amicability / anti-amicability graph}
of the Clifford algebra $\R_{m,m},$
the restriction being the requirement that an edge only exists for pairs of matrices with disjoint support.
\end{definition}

\begin{definition}\label{definition-red-subgraph}
For a graph $\Gamma$ with edges coloured by -1 (red) and 1 (blue),
$\Gamma[-1]$ denotes the \emph{red subgraph} of $\Gamma,$
the graph containing all of the vertices of $\Gamma,$ and all of the red (-1) coloured edges.
Similarly, $\Gamma[1]$ denotes the \emph{blue subgraph} of $\Gamma.$
\end{definition}

The following theorem is proven in the paper on bent functions\cite{Leo15Bent}.

\begin{theorem}\label{theorem-twins-are-strongly-regular}
%
For all $m \geqslant 1,$
both graphs $\varDelta_m[-1]$ and $\varDelta_m[1]$ is strongly regular, with parameters
$v_m = 4^m,$ $k_m = 2^{2 m - 1} - 2^{m - 1},$ $\lambda_m=\mu_m=2^{2 m - 2} - 2^{m - 1}.$
%
\end{theorem}

The existence of a permutation $\pi$ with Property 1 is equivalent to the graph $\varDelta_m$
having the following property.

\begin{propertyonea}
The graph $\varDelta_m$ is self-edge-colour-complementary.
That is, there exists a permutation of the vertices which takes every red edge to a blue edge and vice-versa.
(This permutation is $\pi$ itself.) 
\end{propertyonea}

\section{Main Results}
\begin{Theorem}
\label{Conjectures-are-false-theorem}
For $m \geqslant 1$ the following hold.
\begin{enumerate}
 \item 
There exist transversal graphs that do not have an edge-colour-complement.
Thus Conjecture~\ref{conjecture-3} is false.
\item
As a consequence, Conjectures~\ref{conjecture-1} and~\ref{conjecture-2} are also false.
\item
Question~\ref{Question-1} is resolved.
The only $m \geqslant 1$ for which there an automorphism of $\varDelta_m$ that swaps the subgraphs $\varDelta_m[-1]$ and $\varDelta_m[1]$
are $m=1,2$ and $3$.
\end{enumerate}

\end{Theorem}

\begin{Lemma}
\label{Red-clique-lemma}
The order of the largest clique in the red subgraph $\varDelta_m[-1]$ is at most $\rho(2^m)$,
where $\rho$ is the Hurwitz-Radon function \cite{GerP74,Hur22,Rad22}:
\begin{align*}
\rho(2^{4 d + c}) &:= 2^c + 8 d, \quad \text{where~} 0 \leqslant c < 4.
\end{align*}
(In particular, $\rho(2)=2$, $\rho(4)=4$, $\rho(8)=8$, but $\rho(16)=9$. 
In general, $\rho(2^m) < 2^m$ for $m \geqslant 4$.
\end{Lemma}

\begin{Lemma}
\label{Blue-clique-lemma}
The order of the largest clique in the blue subgraph $\varDelta_m[1]$ is at least $2^m$.
\end{Lemma}
\begin{proof}
We construct a clique of order $2^m$, using the following vertices denoted in base 4:
\begin{align*}
00 &\ldots 02
\\
00 &\ldots 20
\\
&\ldots
\\
22 &\ldots 22
\end{align*}
This set is closed under addition in $\Z_2^{2 m}$,
and therefore forms a clique of order $2^m$ in $\varDelta_m[1]$.
\end{proof}

\begin{proofof}{Theorem~\ref{Conjectures-are-false-theorem}}
Assume that $m \geqslant 4$.
A transversal graph is a subgraph of $\varDelta_m$ which is a complete graph of order $2^m$.
The edges of a transversal graph are labelled with -1 (red) or 1 (blue).
By Lemma~\ref{Red-clique-lemma}, the largest clique of $\varDelta_m[-1]$ is of order $\rho(2^m) < 2^m$,
and by Lemma~\ref{Blue-clique-lemma}, the largest clique of $\varDelta_m[1]$ is of order $2^m$.
If we take a blue clique of order $2^m$ as a transversal graph, this cannot have an edge-colour-complement
in $\varDelta_m$, because no red clique can be this large.
More generally, we need only take a transversal graph containing a blue clique with order larger than $\rho(2^m)$.
This falsifies Conjecture~\ref{conjecture-3}.

Since Conjecture~\ref{conjecture-3} fails for $m \geqslant 4$, 
the pairing of graphs described in Conjecture~\ref{conjecture-2} is impossible for $m \geqslant 4$.
Thus Conjecture~\ref{conjecture-2} is also false.

Finally, Conjecture~\ref{conjecture-1} fails as a direct consequence of Lemmas~\ref{Red-clique-lemma} and~\ref{Blue-clique-lemma},
since, for $m \geqslant 4$, the difference between $\varDelta_m[-1]$  and $\varDelta_m[1]$ in the order of the largest clique
means that these two subgraphs of $\varDelta_m$ cannot be isomorphic.
Therefore, for $m \geqslant 4$,  there can be no automorphism of $\varDelta_m$ that swaps the edge colours. 
\end{proofof}


%%%%%%%%%%%%%%%%%%%%%%%%%%%%%%%%%%%%%%%%%%%%%%%%%%%%%%%%%%%%%%
\subsection*{Acknowledgements.}

Thanks to Christine Leopardi for her hospitality.
Thanks to Robert Craigen, William Martin,
Padraig {\'O} Cath{\'a}in and Judy-anne Osborn for valuable discussions.
This work was begun in 2014 while the author was a Visiting Fellow at the Australian National
University, and concluded while the author was a Visiting Fellow and a Casual Academic at the University of Newcastle, Australia.

%%%%%%%%%%%%%%%%%%%%%%%%%%%%%%%%%%%%%%%%%%%%%%%%%%%%%%%%%%%%%%%%%%%%%%

%\addtocontents{toc}{\vspace{0.5cm}}

%\newpage

\bibliographystyle{abbrv-par}

\bibliography{bib}
% 
% \begin{thebibliography}{10}
% 
% \bibitem{BasHJ09}
% M.~Bastian, S.~Heymann, and M.~Jacomy.
% \newblock Gephi: an open source software for exploring and manipulating
%   networks.
% \newblock In {\em ICWSM}, (2009).
% 
% \bibitem{BerC99}
% A.~Bernasconi and B.~Codenotti.
% \newblock Spectral analysis of {Boolean} functions as a graph eigenvalue
%   problem.
% \newblock {\em IEEE Transactions on Computers}, 48(3):345--351, (1999).
% 
% \bibitem{BerCV01}
% A.~Bernasconi, B.~Codenotti, and J.~M. VanderKam.
% \newblock A characterization of bent functions in terms of strongly regular
%   graphs.
% \newblock {\em IEEE Transactions on Computers}, 50(9):984--985, (2001).
% 
% \bibitem{Bos63}
% R.~C. Bose.
% \newblock Strongly regular graphs, partial geometries and partially balanced
%   designs.
% \newblock {\em Pacific J. Math}, 13(2):389--419, (1963).
% 
% \bibitem{Bra85}
% H.~W. Braden.
% \newblock {$n$}-dimensional spinors: their properties in terms of finite
%   groups.
% \newblock {\em J. Math. Phys.}, 26(4):613--620, (1985).
% 
% \bibitem{Bre11}
% R.~Brent.
% \newblock Private communication.
% \newblock (2011).
% 
% \bibitem{BroCN89}
% A.~E.~Brouwer, A.~Cohen, and A.~Neumaier.
% \newblock {\em Distance-Regular Graphs}.
% \newblock Ergebnisse der Mathematik und Ihrer Grenzgebiete, 3 Folge/A Series of
%   Modern Surveys in Mathematics Series. Springer London, Limited, (2011).
% 
% \bibitem{BroH12}
% A.~E. Brouwer and W.~H. Haemers.
% \newblock {\em Spectra of graphs}.
% \newblock Universitext. Springer, New York, (2012).
% 
% \bibitem{Brov84}
% A.~E. Brouwer and J.~H. van Lint.
% \newblock Strongly regular graphs and partial geometries.
% \newblock {\em Enumeration and design (Waterloo, Ont., 1982)},  85--122,
%   (1984).
% 
% \bibitem{Cra95}
% R.~Craigen.
% \newblock Signed groups, sequences, and the asymptotic existence of {Hadamard}
%   matrices.
% \newblock {\em J. Combin. Theory Ser. A}, 71(2):241--254, (1995).
% 
% \bibitem{Cra11}
% R.~Craigen.
% \newblock A taxonomy of orthogonal matrices.
% \newblock {International Workshop on Hada\-mard Matrices and their
%   Applications, RMIT}, (2011).
% 
% \bibitem{CsaN06}
% G.~Csardi and T.~Nepusz.
% \newblock The igraph software package for complex network research.
% \newblock {\em InterJournal, Complex Systems}, 1695(5), (2006).
% 
% \bibitem{deLF11}
% W.~de~Launey and D.~L. Flannery.
% \newblock {\em Algebraic design theory}.
% \newblock Number 175 in Mathematical Surveys and Monographs. American
%   Mathematical Society, Providence, RI, (2011).
% 
% \bibitem{deLK09}
% W.~de~Launey and H.~Kharaghani.
% \newblock On the asymptotic existence of cocyclic {Hadamard} matrices.
% \newblock {\em Journal of Combinatorial Theory, Series A}, 116(6):1140--1153,
%   (2009).
% 
% \bibitem{deLS01}
% W.~de~Launey and M.~J. Smith.
% \newblock Cocyclic orthogonal designs and the asymptotic existence of cocyclic
%   {Hadamard} matrices and maximal size relative difference sets with forbidden
%   subgroup of size 2.
% \newblock {\em Journal of Combinatorial Theory, Series A}, 93(1):37--92,
%   (2001).
% 
% \bibitem{Dil74}
% J.~F. Dillon.
% \newblock {\em Elementary Hadamard Difference Sets}.
% \newblock PhD thesis, University of Maryland College Park, Ann Arbor, USA,
%   (1974).
% 
% \bibitem{Fla97}
% D.~L. Flannery.
% \newblock Cocyclic {Hadamard} matrices and {Hadamard} groups are equivalent.
% \newblock {\em J. Algebra}, 192(2):749--779, (1997).
% 
% \bibitem{Gan12}
% E.~R. Gansner.
% \newblock Drawing graphs with {Graphviz}.
% \newblock Technical report, AT\&T Bell Laboratories, (2012).
% \newblock http://www.graphviz.org/pdf/oldlibguide.pdf, (accessed 14 October
%   2013).
% 
% \bibitem{Gas80}
% H.~M. Gastineau-Hills.
% \newblock {\em Systems of orthogonal designs and quasi-{Clifford} algebras}.
% \newblock PhD thesis, University of Sydney, (1980).
% 
% \bibitem{Gas82}
% H.~M. Gastineau-Hills.
% \newblock Quasi-{Clifford} algebras and systems of orthogonal designs.
% \newblock {\em J. Austral. Math. Soc. Ser. A}, 32(1):1--23, (1982).
% 
% \bibitem{GerP74}
% A.~V. Geramita and N.~J. Pullman.
% \newblock Radon's function and {Hadamard} arrays.
% \newblock {\em Linear and Multilinear Algebra}, 2:147--150, (1974).
% 
% \bibitem{GerS79}
% A.~V. Geramita and J.~Seberry.
% \newblock {\em Orthogonal designs: quadratic forms and {Hadamard} matrices},
%   volume~45 of {\em Lecture Notes in Pure and Applied Mathematics}.
% \newblock Marcel Dekker Inc., New York, (1979).
% 
% \bibitem{GoeS67}
% J.-M. Goethals and J.~J. Seidel.
% \newblock Orthogonal matrices with zero diagonal.
% \newblock {\em Canad. J. Math.}, 19:1001--1010, (1967).
% 
% \bibitem{HagSS13}
% A.~Hagberg, D.~Schult, and P.~Swart.
% \newblock Networkx reference, (2013).
% \newblock http://networkx.github.io/documentation/latest/reference/index.html,
%   (accessed 14 October 2013).
% 
% \bibitem{HamS81}
% J.~Hammer and J.~Seberry.
% \newblock Higher dimensional orthogonal designs and {Hadamard} matrices.
% \newblock {\em Congressus Numerantium}, 31:95--108, (1981).
% 
% \bibitem{Hor07}
% K.~J. Horadam.
% \newblock {\em Hadamard matrices and their applications}.
% \newblock Princeton University Press, Princeton, NJ, (2007).
% 
% \bibitem{HordeL93}
% K.~J. Horadam and W.~de~Launey.
% \newblock Cocyclic development of designs.
% \newblock {\em J. Algebraic Combin.}, 2(3):267--290, (1993).
% 
% \bibitem{Hur22}
% A.~Hurwitz.
% \newblock {\"Uber} die {Komposition} der quadratischen {Formen}.
% \newblock {\em Math. Ann.}, 88(1-2):1--25, (1922).
% 
% \bibitem{IndG08}
% G.~Indulal and I.~Gutman.
% \newblock On the distance spectra of some graphs.
% \newblock {\em Mathematical communications}, 13(1):123--131, (2008).
% 
% \bibitem{Isa08}
% I.~M. Isaacs.
% \newblock {\em Finite group theory}, volume~92 of {\em Graduate Studies in
%   Mathematics}.
% \newblock American Mathematical Society, Providence, RI, (2008).
% 
% \bibitem{Ito94}
% N.~Ito.
% \newblock On {Hadamard} groups.
% \newblock {\em J. Algebra}, 168(3):981--987, (1994).
% 
% \bibitem{KotK09}
% I.~S. Kotsireas and C.~Koukouvinos.
% \newblock Hadamard matrices of {Williamson} type: A challenge for computer
%   algebra.
% \newblock {\em Journal of Symbolic Computation}, 44(3):271--279, (2009).
% 
% \bibitem{LamS89}
% T.~Y. Lam and T.~Smith.
% \newblock On the {Clifford}-{Littlewood}-{Eckmann} groups: a new look at
%   periodicity mod {$8$}.
% \newblock {\em Rocky Mountain J. Math.}, 19(3):749--786, (1989).
% \newblock Quadratic forms and real algebraic geometry (Corvallis, OR, 1986).
% 
% \bibitem{Leo05}
% P.~Leopardi.
% \newblock A generalized {FFT} for {Clifford} algebras.
% \newblock {\em Bull. Belg. Math. Soc. Simon Stevin}, 11(5):663--688, (2004).
% 
% \bibitem{LeoANU}
% P.~Leopardi.
% \newblock Personal home page.
% \newblock {http://maths.anu.edu.au/$\sim$leopardi/}, (2013).
% 
% \bibitem{Lou97}
% P.~Lounesto.
% \newblock {\em Clifford algebras and spinors}, volume 239 of {\em London
%   Mathematical Society Lecture Note Series}.
% \newblock Cambridge University Press, Cambridge, (1997).
% 
% \bibitem{Mac74}
% E.~C. MacRae.
% \newblock Matrix derivatives with an application to an adaptive linear decision
%   problem.
% \newblock {\em Ann. Statist.}, 2:337--346, (1974).
% 
% \bibitem{Men62}
% P.~K. Menon.
% \newblock On difference sets whose parameters satisfy a certain relation.
% \newblock {\em Proceedings of the American Mathematical Society},
%   13(5):739--745, (1962).
% 
% \bibitem{OCat08}
% P.~{\'O}~Cath{\'a}in.
% \newblock Group actions on {Hadamard} matrices.
% \newblock Master's thesis, The National University of Ireland, Galway,
%   Albuquerque, New Mexico, October 2008.
% 
% \bibitem{OCat11}
% P.~{\'O}~Cath{\'a}in.
% \newblock {\em Automorphisms of Pairwise Combinatorial Designs}.
% \newblock PhD thesis, The National University of Ireland, Galway, Albuquerque,
%   New Mexico, December 2011.
% 
% \bibitem{Osb11}
% J.~Osborn.
% \newblock On small-order {Hadamard} matrices from the {Williamson} and octonion
%   constructions.
% \newblock Presented at 35ACCMCC, Monash Univeristy, (2011).
% 
% \bibitem{Por69}
% I.~R. Porteous.
% \newblock {\em Topological geometry}.
% \newblock Van Nostrand Reinhold Co., London, (1969).
% 
% \bibitem{Rad22}
% J.~Radon.
% \newblock Lineare {Scharen} orthogonaler {Matrizen}.
% \newblock {\em Abhandlungen aus dem Mathematischen Seminar der Universit{\"a}t
%   Hamburg}, 1(1):1--14, (1922).
% 
% \bibitem{Rob77}
% P.~J. Robinson.
% \newblock Using product designs to construct orthogonal designs.
% \newblock {\em Bulletin of the Australian Mathematical Society},
%   16(02):297--305, (1977).
% 
% \bibitem{Rot76}
% O.~S. Rothaus.
% \newblock On ``bent'' functions.
% \newblock {\em Journal of Combinatorial Theory, Series A}, 20(3):300--305,
%   (1976).
% 
% \bibitem{SebY92}
% J.~Seberry and M.~Yamada.
% \newblock Hadamard matrices, sequences, and block designs.
% \newblock In J.~Dinitz and D.~Stinson, editors, {\em Contemporary Design
%   Theory: A Collection of Surveys}, Wiley Interscience Series in Discrete
%   Mathematics. Wiley, (1992).
% 
% \bibitem{Shr59}
% S.~S. Shrikhande.
% \newblock The uniqueness of the l\_2 association scheme.
% \newblock {\em The annals of mathematical statistics},  781--798, (1959).
% 
% \bibitem{Slo13}
% N.~J.~A. Sloane.
% \newblock The on-line encyclopedia of integer sequences.
% \newblock http://oeis.org, (accessed 14 October 2013).
% 
% \bibitem{Tau71}
% O.~Taussky.
% \newblock {$(1,\,2,\,4\,,8)$}-sums of squares and {Hadamard} matrices.
% \newblock In {\em Combinatorics ({Proc}. {Sympos}. {Pure} {Math}., {Vol}.
%   {XIX}, {Univ}. {California}, {Los} {Angeles}, {Calif}., 1968)},  229--233.
%   Amer. Math. Soc., Providence, R.I., (1971).
% 
% \bibitem{Wil44}
% J.~Williamson.
% \newblock Hadamard's determinant theorem and the sum of four squares.
% \newblock {\em Duke Math. J.}, 11:65--81, (1944).
% 
% \bibitem{Wol74}
% W.~W. Wolfe.
% \newblock Clifford algebras and amicable orthogonal designs.
% \newblock Queen's Mathematical Preprint 1974-22, Queen's University, Kingston,
%   Ontario, (1974).
% 
% \end{thebibliography}

\end{document}
% ----------------------------------------------------------------
