\documentclass[pdf,sprung,slideColor,nocolorBG]{beamer}
%
%\documentclass[hyperref={pdfpagelabels=false}]{beamer}
\mode<presentation>

\let\Tiny=\tiny
\usetheme{Adelaide}
\usefonttheme[stillsansseriftext]{serif}
\setbeamerfont{structure}{series=\bfseries}
\setbeamertemplate{frametitle}[default][center]
\usepackage[latin1]{inputenc}
\usepackage{amsmath}   %needed for \begin{align}... \end{align} environment
\usepackage{amsfonts}
\usepackage{amssymb}
\usepackage{amscd}
\usepackage[all]{xy}
\usepackage{xcolor}
\usepackage{enumerate}
%
\newcommand{\slidecite}[1]{\tiny{(#1)}\normalsize{}}
\newcommand{\smallcite}[1]{\small{(#1)}\normalsize{}}

\newcommand{\mb}[1]{\mathbb{#1}}
\newcommand{\mf}[1]{\mathbf{#1}}
\newcommand{\Emph}[1]{\emph{\textcolor{blue}{#1}}}

\newcommand{\abs}[1]{\left| #1 \right|}
\newcommand{\norm}[1]{\left\| #1 \right\|}
\newcommand{\To}{\rightarrow}

\newcommand{\E}{\mf{\operatorname{e}}}

\newcommand{\Cay}[1]{\operatorname{Cay}\left(#1\right)}
\newcommand{\Clique}[1]{\omega\left(#1\right)}
\newcommand{\dual}[1]{\widetilde{#1}}

\newcommand{\G}{\mb{G}}
\newcommand{\R}{\mb{R}}
\newcommand{\Z}{\mb{Z}}
%\newtheorem{Definition}{Definition}
\newtheorem{Question}{Question}
\newtheorem{Conjecture}{Conjecture}

\title{Classifying bent functions by their Cayley graphs}
\author{Paul Leopardi}

\date{8 December 2015}

\institute{School of Mathematical and Physical Sciences, University of Newcastle.}
\titlegraphic{
%\includegraphics[angle=0,width=10mm]{../../common/beamer-anu-colourlogo.png}
\includegraphics[angle=0,width=20mm]{../../common/carma_logo.jpg}
}
\begin{document}

\frame{\titlepage}
\begin{frame}
\frametitle{Acknowledgements}
\begin{center}
~
 
Robert Craigen, William Martin,
Padraig {\'O} Cath{\'a}in and 

Judy-anne Osborn.

~

Australian National University.

~

SageMathCloud.

~

Australian Bureau of Meteorology.

~

The Russell family.
\end{center}
\end{frame}

\begin{frame}
\frametitle{Overview}
%\begin{center}
\begin{itemize}
\item
Key concepts.

~

\item
Equivalence.

~

\item
Some results.

~

\item
Observations for small dimensions.

~

\item
Some questions.

~

\item
SageMathCloud worksheet.
\end{itemize}
 
%\end{center}
\end{frame}

\section{Key concepts}

\begin{frame}
\frametitle{The Cayley graph of a binary function}
%\begin{center}
The \Emph{Cayley graph} $\Cay{f}$ of a binary function 

~

\begin{align*}
%
f : \Z_2^n \To \Z_2 \quad \text{where} \quad f(0) = 0 
% 
\end{align*}

~

is 
an undirected graph with 

\begin{align*}
V(\Cay{f}) &:= \Z_2^n, \quad (x,y) \in E(\Cay{f}) \Leftrightarrow f(x+y) = 1.
\end{align*}

~

\slidecite{Bernasconi and Codenotti 1999}
\end{frame}
\begin{frame}
\frametitle{Bent functions}

A binary function  $f : \Z_2^{2m} \To \Z_2$ is \Emph{bent} if and only if the function $\dual{f}$, defined by
\begin{align*}
(-1)^{\dual{f}(x)} &:= 2^{-m} \sum_{y \in \Z_2^{2m}} (-1)^{f(y) + \langle x, y \rangle}
\end{align*}
is a binary function on $\Z_2^{2m}$.

~

The function $\dual{f}$ is also bent and is called the \Emph{dual} of $f$.

~

\slidecite{Dillon 1974; Rothaus 1976; Tokareva 2011}
\end{frame}

\begin{frame}
\frametitle{Strongly regular graphs}
%\begin{center}
A simple graph $\Gamma$ of order $v$ is \Emph{strongly regular} with parameters 
$(v,k,\lambda,\mu)$ if 

~

\begin{itemize}
 \item 
each vertex has degree $k,$ 

~
 \item 
each adjacent pair of vertices has $\lambda$ common neighbours, and

~
\item
each nonadjacent pair of vertices has $\mu$ common neighbours.
\end{itemize}

~

\slidecite{Brouwer, Cohen and Neumaier 1989}

%\end{center}
\end{frame}

\begin{frame}
\frametitle{Bent functions and strongly regular graphs}

\begin{Theorem}
\smallcite{Bernasconi and Codenotti 1999}

The Cayley graph $\Cay{f}$ of a bent function $f$ on $\Z_2^{2m}$ 

(with $f(0)=0$) is a strongly regular graph with $\lambda = \mu.$ 
\end{Theorem}

The parameters of $\Cay{f}$ are
\begin{align*}
(v,k,\lambda) = &(4^m, 2^{2 m - 1} - 2^{m-1}, 2^{2 m - 2} - 2^{m-1}) 
\\
  \text{or} \quad &(4^m, 2^{2 m - 1} + 2^{m-1}, 2^{2 m - 2} + 2^{m-1}).
\end{align*}

~

\slidecite{Menon 1962; Dillon 1974; Bernasconi and Codenotti 1999}
%\end{center}
\end{frame}
\begin{frame}
\frametitle{Projective two-weight binary codes}

\begin{Definition}
A \Emph{two-weight binary code} with parameters $[n,k,d]$ is a $k$ dimensional subspace of $\mathbb{F}_2^n$ with 
minimum Hamming distance $d$, such that the set of Hamming weights of the non-zero vectors has size 2.

~

``A \Emph{generator matrix} $G$ of a linear code $[n, k]$ code $C$ is any matrix
of rank $k$ (over $\mathbb{F}_2$) with rows from $C.$''

~

``A linear $[n, k]$ code is called \Emph{projective} if no two columns of a generator matrix
$G$ are linearly dependent, i.e., if the columns of $G$ are pairwise different points in a
projective $(k-1)$-dimensional space.''
In the case of $\mathbb{F}_2$, no two columns are equal.

~

\slidecite{Bouyukliev, Fack, Willems and Winne 2006}

\end{Definition}

\end{frame}
\section{Equivalence}
\begin{frame}
\frametitle{Cayley equivalence}
\begin{Definition}
%
For $f, g : \Z_2^{2m} \To \Z_2$, with both $f$ and $g$ bent, 

we call $f$ and $g$ \Emph{Cayley equivalent},
and write $f \equiv g$, 

if and only if $f(0)=g(0)=0$ and $\Cay{f} \equiv \Cay{g}$ as graphs.

~

Equivalently, $f \equiv g$ if and only if $f(0)=g(0)=0$ and 

there exists a bijection $\pi : \Z_2^{2m} \To \Z_2^{2m}$ such that
\begin{align*}
g(x+y) &= f \big(\pi(x)+\pi(y)\big) \quad \text{for all~} x,y \in \Z_2^{2m}. 
\end{align*}
\end{Definition}
\end{frame}
\begin{frame}
\frametitle{Extended Cayley equivalence}
\begin{Definition}
For $f, g : \Z_2^{2m} \To \Z_2$, with both $f$ and $g$ bent,

if there exist $\delta, \epsilon \in \{0,1\}$ such that $f + \delta \equiv g + \epsilon$, 

we call $f$ and $g$ \Emph{extended Cayley (EC) equivalent} and write $f \cong g$. 
\end{Definition}
Extended Cayley equivalence is an equivalence relation on the set of all bent functions on $\Z_2^{2m}$.
\end{frame}
\section{Some results}
\begin{frame}
\frametitle{Two infinite sequences: definitions}

For $m>0$, the functions $\sigma_m, \tau_m : \Z_2^{2m} \To \Z_2$ are defined by mapping $\Z_2^{2m}$ to $\Z_{2^m}$ in lex order, 
expanding in base 4, and counting the digits 0, 1, 2, 3 in the expansion.

For $d \in \{0,1,2,3\}, x \in \Z_2^{2m}$, call this count $\sharp(d,x)$ 

~

$\sigma_m(x) = 1$ if and only $\sharp(1,x)$ is odd.

$\tau_m(x) = 1$ if and only if $\sharp(1,x) + \sharp(2,x) > 0$ and $\sharp(1,x)$ is even.

~

Both $\sigma_m$  and $\tau_m$ are bent functions, with $\sigma_m(0)=\tau_m(0)=0.$

~

Both $\Cay{\sigma_m}$  and $\Cay{\tau_m}$ have parameters 
$(v,k,\lambda) = (4^m, 2^{2 m - 1} - 2^{m-1}, 2^{2 m - 2} - 2^{m-1})$.

~

\slidecite{L 2014, 2015}
\end{frame}
\begin{frame}
\frametitle{Two infinite sequences: clique numbers}

\begin{theorem}
$\sigma_m \equiv \tau_m$ for $m=1,2,3$, but $\sigma_m \ncong \tau_m$ for $m > 3$:

~

If $\Clique{f}$ is the \Emph{clique number} of $\Cay{f}$, then
\begin{align*}
\Clique{\sigma_m} &= \rho(2^m) \text{~and~}
\Clique{\tau_m} = 2^m,
\end{align*}
where $\rho$ is the \Emph{Hurwitz-Radon function}, defined by
\begin{align*}
\rho(2^{4 d + c}) &:= 2^c + 8 d, \quad \text{for~} 0 \leqslant c < 4.
\end{align*}
 
\end{theorem}

~
 
\slidecite{L 2015}
\end{frame}

\begin{frame}
\frametitle{Linear equivalence implies Cayley equivalence}

\begin{Theorem}
If $f$ is bent with $f(0)=0$ and $g(x) := f(A x)$ where $A \in GL(2m,2)$,
then $g$ is bent with $g(0)=0$ and $f \equiv g$.
\end{Theorem}
\begin{proof}
\begin{align*}
g(x+y) &= f\big(A(x+y)\big) = f(A x + A y)\quad \text{for all~} x,y \in \Z_2^{2m}. 
\end{align*}
\end{proof}
 
\end{frame}

\begin{frame}
\frametitle{Extended affine equivalence (1)}

\begin{Definition}
For bent functions $f,g : \Z_2^{2m} \To \Z_2$, 

$f$ is \Emph{extended affine equivalent} to $g$ if and only if
\begin{align*}
g(x) &= f(A x + b) + \langle c, x \rangle + \delta 
\end{align*}
for some $A \in GL(2m,2)$, $b, c \in \Z_2^{2m}$, $\delta \in \Z_2$.
\end{Definition}
~

\slidecite{Tokareva 2014}
\end{frame}

\begin{frame}
\frametitle{Extended affine equivalence (2)}

\begin{Theorem}
For $A \in GL(2m,2)$, $b, c \in \Z_2^{2m}$, $\delta \in \Z_2$,
$f : \Z_2^{2m} \To \Z_2$, 

the function
\begin{align*}
h(x) &:= f(A x + b) + \langle c, x \rangle + \delta
\intertext{can be expressed as $h(x) = g(A x)$ where}
g(x) &:= f(x+b) + \langle (A^{-1})^T c, x \rangle + \delta,
\end{align*}
and therefore if $f$ is bent and $h(0)=0$ then $h \equiv g$.
\end{Theorem}
\end{frame}

\begin{frame}
\frametitle{Extended affine equivalence (3)}

Therefore, to determine the extended Cayley equivalence classes within the extended affine equivalence class of
a bent function $f : \Z_2^{2m} \To \Z_2$, we need only examine functions of the form
\begin{align*}
f(x+b) + \langle c, x \rangle + f(b),
\end{align*}
for each $b, c \in \Z_2^{2m}$.
\end{frame}
\begin{frame}
\frametitle{Dual functions (1)}
\begin{Theorem}
\smallcite{Carlet 2007, Proposition 4} 

~

For a bent function $f : \Z_2^{2m} \To \Z_2$, and $b,c \in \Z_2^{2m}$,
if
\begin{align*}
g(x) &:= f(x+b) + \langle c, x \rangle
\intertext{then}
\dual{g}(x) &= \dual{f}(x+c) + \langle b, x \rangle + \langle b, c \rangle. 
\end{align*}
\end{Theorem}
 
\end{frame}
\begin{frame}
\frametitle{Dual functions (2)}
\begin{Theorem}
For a bent function $f : \Z_2^{2m} \To \Z_2$, and $A \in GL(2 m, 2)$,
if
\begin{align*}
g(x) &:= f(A x)
\intertext{then}
\dual{g}(x) &= \dual{f}\big((A^T)^{-1} x \big),
\end{align*}
\end{Theorem}
and therefore $\dual{g} \equiv \dual{f}$.
If, in addition, $f=\dual{f}$ then $\dual{g} \equiv g$. 
 
\end{frame}
\begin{frame}
\frametitle{Dual functions (3)}

Functions of the form 
\begin{align*}
f(x) := \sum_{k=0}^{m-1} x_{2k} x_{2k+1}
\end{align*}
are self dual bent functions, $f=\dual{f}$.

~

There are many other self dual bent functions.

~

\slidecite{Carlet, Danielson, Parker and Sol\'e 2008; Feulner, Sok, Sol\'e and Wassermann 2011}
\end{frame}

\begin{frame}
\frametitle{The two block designs of a bent function}

The adjacency matrix of $\Cay{f}$ can also be interpreted as the incidence matrix of a block design.

~

In this case we do not need $f(0)=0$.

~

A second block design described by Dillon and Schatz can be defined by the incidence matrix $D(f)$ where
\begin{align*}
D(f)_{c,x} &:= f(x) + \langle c, x \rangle + \dual{f}(c).   
\end{align*}
This is a symmetric block design with the \emph{symmetric difference property}.

~

\slidecite{Dillon and Schatz 1987; Neumann 2006}
\end{frame}
\section{Observations}
\begin{frame}
\frametitle{For $m=1$}

One extended affine class: $[f_{2,1}]$ 

where $f_{2,1}(x) := x_0 x_1$ is self dual.

~

Two extended Cayley classes:
\begin{align*}
\begin{array}{|cccl|}
\hline
\text{Class} &
\text{Parameters} & 
\text{2-rank} &
\text{Clique polynomial}
\\
\hline
1 &
(4, 1, 0, 0) & 4 & 
2t^{2} + 4t + 1
\\
2 &
K_4 & 4 & 
t^{4} + 4t^{3} + 6t^{2} + 4t + 1
\\
\hline
\end{array}
\end{align*}

\end{frame}
\begin{frame}
\frametitle{For $m=2$: classes}

One extended affine class: $[f_{4,1}]$ where 

$f_{4,1}(x) := x_0 x_1 + x_2 x_3$ is self dual.

~

Two extended Cayley classes:
\begin{align*}
\begin{array}{|cccl|}
\hline
\text{Class} &
\text{Parameters} & 
\text{2-rank} &
\text{Clique polynomial}
\\
\hline
1 &
(16, 6, 2, 2) & 
6 &
8t^{4} + 32t^{3} + 48t^{2} + 16t + 1
\\
2 &
(16, 10, 6, 6) & 
6 &
\begin{array}{l}
16t^{5} + 120t^{4} + 160t^{3} + 
\\
80t^{2} + 16t + 1
\end{array}
\\
\hline
\end{array}
\end{align*}
\end{frame}
\begin{frame}
\frametitle{For $m=2$: two-weight codes}

The Cayley graphs for classes 1 and 2 are isomorphic to those those obtained from the following two-weight projective
codes:

\begin{align*}
\begin{array}{|ccc|}
\hline
\text{Class} &
\text{Parameters} & \text{Generator matrix}
\\
\hline
1 &
[6, 4, 2] & 
\left[
\begin{array}{cccccc}
0 & 0 & 1 & 1 & 1 & 1
\\
1 & 0 & 0 & 1 & 1 & 1
\\
1 & 1 & 1 & 1 & 0 & 0
\\
0 & 1 & 1 & 1 & 1 & 0
\end{array}
\right]
\\
2 &
[5, 4, 2] & 
\left[
\begin{array}{ccccc}
1 & 1 & 0 & 0 & 0
\\
0 & 1 & 1 & 0 & 0
\\
0 & 0 & 0 & 1 & 1
\\
1 & 0 & 0 & 0 & 1
\end{array}
\right]
\\
\hline
\end{array}
\end{align*}

\end{frame}
\begin{frame}
\frametitle{For $m=3$: extended affine classes}

Four extended affine classes:

\begin{align*}
\def\arraystretch{1.2}
\begin{array}{|cl|}
\hline
\text{Class} &
\text{Representative}
\\
\hline
\,[f_{6,1}] & f_{6,1} := 
\begin{array}{l}
x_{0} x_{1} + x_{2} x_{3} + x_{4} x_{5}
\end{array}
\\
\,[f_{6,2}] & f_{6,2} := 
\begin{array}{l}
x_{0} x_{1} x_{2} + x_{0} x_{3} + x_{1} x_{4} + x_{2} x_{5}
\end{array}
\\
\,[f_{6,3}] & f_{6,3} := 
\begin{array}{l}
x_{0} x_{1} x_{2} + x_{0} x_{1} + x_{0} x_{3} + x_{1} x_{3} x_{4} + x_{1} x_{5} + 
\\
x_{2} x_{4} + x_{3} x_{4}
\end{array}
\\
\,[f_{6,4}] & f_{6,4} := 
\begin{array}{l}
x_{0} x_{1} x_{2} + x_{0} x_{3} + x_{1} x_{3} x_{4} + x_{1} x_{5} + x_{2} x_{3} x_{5} + 
\\
x_{2} x_{3} + x_{2} x_{4} + x_{2} x_{5} + x_{3} x_{4} + x_{3} x_{5}
\end{array}
\\
\hline
\end{array}
\end{align*}
\end{frame}
\begin{frame}
\frametitle{For EA class $[f_{6,1}]$: classes}

The function
$f_{6,1}(x) = x_0 x_1 + x_2 x_3 + x_4 x_5$ is self dual.

~

Two extended Cayley classes:
\small{}
\begin{align*}
\def\arraystretch{1.2}
\begin{array}{|cccl|}
\hline
\text{Class} &
\text{Parameters} & 
\text{2-rank} &
\text{Clique polynomial}
\\
\hline
1 &
(64, 28, 12, 12) & 8 & 
\begin{array}{l}
64t^{8} + 512t^{7} + 1792t^{6} + 3584t^{5} +
\\
5376t^{4} + 3584t^{3} + 896t^{2} + 64t + 1
\end{array}
\\
2 &
(64, 36, 20, 20) & 8 & 
\begin{array}{l}
2304t^{6} + 13824t^{5} + 19200t^{4} + 
\\
7680t^{3} + 1152t^{2} + 64t + 1
\end{array}
\\
\hline
\end{array}
\end{align*}
\end{frame}
\begin{frame}
\frametitle{For EA class $[f_{6,1}]$: two-weight codes}

The Cayley graphs for classes 1 and 2 are isomorphic to those those obtained from the following two-weight projective
codes as listed by Tonchev (2006):

\begin{align*}
\begin{array}{|ccl|}
\hline
\text{Class} &
\text{Parameters} & \text{Reference}
\\
\hline
1 & [35,6,16] & \text{Tonchev Table 1.56 1, 2 }
\\
2 & [27,6,12] & \text{Tonchev Table 1.55 1 }
\\
\hline
\end{array}
\end{align*}

\slidecite{Tonchev 1996, 2006}
\end{frame}
\begin{frame}
\frametitle{For EA class $[f_{6,2}]$: classes}

The function
$f_{6,2}(x) = x_{0} x_{1} x_{2} + x_{0} x_{3} + x_{1} x_{4} + x_{2} x_{5}$.

~

Three extended Cayley classes:
\small{}
\begin{align*}
\def\arraystretch{1.2}
\begin{array}{|cccl|}
\hline
\text{Class} &
\text{Parameters} & 
\text{2-rank} &
\text{Clique polynomial}
\\
\hline
1 &
(64, 28, 12, 12) & 8 &
\begin{array}{l}
64t^{8} + 512t^{7} + 1792t^{6} + 3584t^{5} + 
\\
5376t^{4} + 3584t^{3} + 896t^{2} + 64t + 1
\end{array}
\\
2 &
(64, 28, 12, 12) & 8 & 
\begin{array}{l}
256t^{6} + 1536t^{5} + 4352t^{4} + 
\\
3584t^{3} + 896t^{2} + 64t + 1
\end{array}
\\
3 &
(64, 36, 20, 20) & 8 &
\begin{array}{l}
192t^{8} + 1536t^{7} + 8960t^{6} + 19968t^{5} +
\\
20224t^{4} + 7680t^{3} + 1152t^{2} + 64t + 1
\end{array}
\\
\hline
\end{array}
\end{align*}
Graph 1 is isomorphic to graph 1 of EA class $[f_{6,1}]$, and is also isomorphic to the complement of Royle's $(64,35,18,20)$ SRG $X$.

~

\slidecite{Royle 2008}
\end{frame}
\begin{frame}
\frametitle{For EA class $[f_{6,2}]$: two-weight codes}

The Cayley graphs for classes 1 to 3 are isomorphic to those those obtained from the following two-weight projective
codes as listed by Tonchev (2006):

\begin{align*}
\begin{array}{|ccl|}
\hline
\text{Class} &
\text{Parameters} & \text{Reference}
\\
\hline
1 & [35,6,16] & \text{Tonchev Table 1.56 1, 2 }
\\
2 & [35,6,16] & \text{Tonchev Table 1.56 3 }
\\
3 & [27,6,12] & \text{Tonchev Table 1.55 2 }
\\
\hline
\end{array}
\end{align*}

\slidecite{Tonchev 1996, 2006}
\end{frame}
\begin{frame}
\frametitle{For EA class $[f_{6,3}]$: classes}

The function
\begin{align*}
f_{6,3}(x) &= x_{0} x_{1} x_{2} + x_{0} x_{1} + x_{0} x_{3} + x_{1} x_{3} x_{4} 
\\
           &+ x_{1} x_{5} + x_{2} x_{4} + x_{3} x_{4}.
\end{align*}

Four extended Cayley classes:
\small{}
\begin{align*}
\def\arraystretch{1.2}
\begin{array}{|cccl|}
\hline
\text{Class} &
\text{Parameters} & 
\text{2-rank} &
\text{Clique polynomial}
\\
\hline
1 &
(64, 28, 12, 12) & 12 & 
\begin{array}{l}
32t^{8} + 256t^{7} + 896t^{6} + 2048t^{5} +
\\
4608t^{4} + 3584t^{3} + 896t^{2} + 64t + 1
\end{array}
\\
2 &
(64, 28, 12, 12) & 12 & 
\begin{array}{l}
64t^{6} + 1024t^{5} + 4096t^{4} + 
\\
3584t^{3} + 896t^{2} + 64t + 1
\end{array}
\\
3 &
(64, 36, 20, 20) & 12 & 
\begin{array}{l}
160t^{8} + 1280t^{7} + 9344t^{6} + 21504t^{5} +
\\
20480t^{4} + 7680t^{3} + 1152t^{2} + 64t + 1
\end{array}
\\
4 &
(64, 36, 20, 20) & 12 & 
\begin{array}{l}
160t^{8} + 1664t^{7} + 9792t^{6} + 21504t^{5} +
\\
20480t^{4} + 7680t^{3} + 1152t^{2} + 64t + 1
\end{array}
\\
\hline
\end{array}
\end{align*}
\end{frame}
\begin{frame}
\frametitle{For EA class $[f_{6,3}]$: two-weight codes}

The Cayley graphs for classes 1 to 4 are isomorphic to those those obtained from the following two-weight projective
codes as listed by Tonchev (2006):

\begin{align*}
\begin{array}{|ccl|}
\hline
\text{Class} &
\text{Parameters} & \text{Reference}
\\
\hline
1 & [35,6,16] & \text{Tonchev Table 1.56 4 }
\\
2 & [35,6,16] & \text{Tonchev Table 1.56 5 }
\\
3 & [27,6,12] & \text{Tonchev Table 1.55 3 }
\\
4 & [27,6,12] & \text{Tonchev Table 1.55 4 }
\\
\hline
\end{array}
\end{align*}

\slidecite{Tonchev 1996, 2006}
\end{frame}
\begin{frame}
\frametitle{For EA class $[f_{6,4}]$: classes}

The function
\begin{align*}
f_{6,4}(x) &= x_{0} x_{1} x_{2} + x_{0} x_{3} + x_{1} x_{3} x_{4} + x_{1} x_{5} + x_{2} x_{3} x_{5} 
\\
           &+ x_{2} x_{3} + x_{2} x_{4} + x_{2} x_{5} + x_{3} x_{4} + x_{3} x_{5}.
\end{align*}

Three extended Cayley classes:
\small{}
\begin{align*}
\def\arraystretch{1.2}
\begin{array}{|cccl|}
\hline
\text{Class} &
\text{Parameters} & 
\text{2-rank} &
\text{Clique polynomial}
\\
\hline
1 &
(64, 28, 12, 12) & 14 &
\begin{array}{l}
32t^{8} + 256t^{7} + 896t^{6} + 1792t^{5} +
\\
4480t^{4} + 3584t^{3} + 896t^{2} + 64t + 1
\end{array}
\\
2 &
(64, 28, 12, 12) & 14 &
\begin{array}{l}
16t^{8} + 128t^{7} + 448t^{6} + 1280t^{5} +
\\
4224t^{4} + 3584t^{3} + 896t^{2} + 64t + 1
\end{array}
\\
3 &
(64, 36, 20, 20) & 14 &
\begin{array}{l}
176t^{8} + 1408t^{7} + 9664t^{6} + 22272t^{5} +
\\
20608t^{4} + 7680t^{3} + 1152t^{2} + 64t + 1
\end{array}
\\
\hline
\end{array}
\end{align*}
\end{frame}
\begin{frame}
\frametitle{For EA class $[f_{6,4}]$: two-weight codes}

The Cayley graphs for classes 1 to 3 are isomorphic to those those obtained from the following two-weight projective
codes as listed by Tonchev (2006):

\begin{align*}
\begin{array}{|ccl|}
\hline
\text{Class} &
\text{Parameters} & \text{Reference}
\\
\hline
1 & [35,6,16] & \text{Tonchev Table 1.56 7 }
\\
2 & [35,6,16] & \text{Tonchev Table 1.56 6 }
\\
3 & [27,6,12] & \text{Tonchev Table 1.55 5 }
\\
\hline
\end{array}
\end{align*}

\slidecite{Tonchev 1996, 2006}
\end{frame}
\section{Questions}
\begin{frame}
\frametitle{Questions (1)}
(Preceded by the values of $m$ for which the question is settled.)
 
\begin{description}
\item[1-3]
How many EC classes are there for each $m$? ``Exponential numbers'' of classes? 
\item[1-3]
Is the number of EC classes within an EA class bounded? What is the bound?
\item[1-4]
Is the number of EC classes within the quadratic EA class always 2?
\item[1-3]
Which EC classes overlap more than one EA class?
\end{description}

\slidecite{Kantor 1983; Jungnickel and Tonchev 1991}
\end{frame}
\begin{frame}
\frametitle{Questions (2)} 

\begin{description}
\item[1-3]
Which bent functions are Cayley equivalent to their dual?
\item[1-3]
Which EC classes contain a self-dual bent function?
\item[2-3]
Which EC classes correspond to projective two-weight codes?
\item[1-3]
For each EA representative $f$, what is the relationship between the Dillon-Schatz SDP incidence matrix
\begin{align*}
D(f)_{c,x} &= f(x) + \langle c, x \rangle + \dual{f}(c)
\intertext{and the matrix of EC classes}
M_{c,b} &= \big[\Cay{x \mapsto f(x+b) + \langle c, x \rangle + f(b)}\big] ?
\end{align*}
\end{description}
\end{frame}
\section{SageMathCloud}
\begin{frame}[fragile]
\frametitle{Public worksheet on SageMathCloud}
~

See
\begin{verbatim}
http://tinyurl.com/jnchhev
\end{verbatim}
 
\end{frame}

\end{document}
