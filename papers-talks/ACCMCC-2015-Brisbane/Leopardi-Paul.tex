%This is an Abstract template for the 39ACCMCC.
\documentclass[12pt]{article}
\pagestyle{empty}
\usepackage{amsmath, amssymb}
% BEGIN EXTRA PACKAGES (if needed, one per line)

% END EXTRA PACKAGES

% any definitions needed, put them here
% \newcommand{\iso}{\cong}

% Abstract environment, for automagic parsing.
\newenvironment{TalkAbstract}{\noindent\ignorespaces}{}

% Please fill in your identification details and title in the lines below. If you have co-authors, uncomment and fill in after \TalkJoint.
\newcommand{\FirstName}{Paul}
\newcommand{\LastName}{Leopardi}
\newcommand{\Uni}{University of Newcastle} % e.g. The University of Melbourne
\newcommand{\email}{paul.leopardi@gmail.com}
\newcommand{\TalkTitle}{Classifying bent functions by their Cayley graphs}
%\newcommand{\TalkJoint}{insert your co-authors, if any}
\newcommand{\mb}[1]{\mathbb{#1}}
\newcommand{\mf}[1]{\mathbf{#1}}
\newcommand{\Cay}{\operatorname{Cay}}
\newcommand{\oE}{\mf{\operatorname{E}}}
\newcommand{\G}{\mb{G}}
\newcommand{\R}{\mb{R}}
\newcommand{\Z}{\mb{Z}}
\newcommand{\Rep}{P}
\newcommand{\V}[1]{\underline{#1}}
\newcommand{\abs}[1]{\left| #1 \right|}
\newcommand{\isomorphic}{\simeq}
\newcommand{\To}{\rightarrow}

\begin{document}
\begin{center}
% DO NOT EDIT ANY OF THE NEXT 12 LINES.
% YOUR TITLE, NAME, EMAIL ADDRESS, AND INSTITUTION
% SHOULD ALL BE ENTERED ABOVE.
 {\Large \scshape  \TalkTitle} \\[3mm]
 \textbf{\FirstName~\LastName} \\
 \texttt{\email} \\[3mm]
 \Uni \\[3mm]
\ifx\TalkJoint\emtpy
\relax
\else
(Joint work with \TalkJoint)\\[3mm]
\fi
\end{center}

\begin{TalkAbstract}
It is well known~\cite{BerC99} that if a bent function $f: \Z_2^{2m} \To \Z_2$ has $f(0)=0$, then it has a strongly regular Cayley graph whose parameters $(v_m,k_m,\lambda_m,\lambda_m)$ depend only on $m$:
\begin{align*}
(v_m,k_m,\lambda_m) &= (4^m, 2^{2 m - 1} \pm 2^{m-1}, 2^{2 m - 2} \pm 2^{m-1}).
\end{align*}
It is perhaps less well known that even if two such Cayley graphs have the same strongly regular graph parameters, they are not necessarily isomorphic.
This talk examines the concepts of \emph{Cayley equivalence} and \emph{extended Cayley equivalence} of bent functions, and compares these equivalence relations to the
better known concepts of affine equivalence and extended affine equivalence.
The relationship between two-weight codes, bent functions and strongly regular graphs is also touched on.
\end{TalkAbstract}

\begin{thebibliography}{1}
\bibitem{BerC99}
A.~Bernasconi and B.~Codenotti.
\newblock Spectral analysis of {Boolean} functions as a graph eigenvalue
  problem.
\newblock {\em IEEE Transactions on Computers}, 48(3):345--351, (1999).
\end{thebibliography}
\end{document}
