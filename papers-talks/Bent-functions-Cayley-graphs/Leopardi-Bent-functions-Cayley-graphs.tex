\documentclass[12pt,a4paper]{article}
%
\usepackage{amsmath}
\usepackage{amssymb}
\usepackage{amsthm}
\usepackage{graphicx}
%
\usepackage[colorlinks=true,allcolors=blue]{hyperref}
%\usepackage{url}
%
\newcommand{\mb}[1]{\mathbb{#1}}
\newcommand{\mf}[1]{\mathbf{#1}}
\newcommand{\G}{\mb{G}}
\newcommand{\R}{\mb{R}}
\newcommand{\Z}{\mb{Z}}
\newcommand{\abs}[1]{\left| #1 \right|}
\newcommand{\norm}[1]{\left\| #1 \right\|}
\newcommand{\isomorphic}{\simeq}
\newcommand{\To}{\rightarrow}
%
%
\newcommand{\slidecite}[1]{\tiny{(#1)}\normalsize{}}
\newcommand{\smallcite}[1]{\small{(#1)}\normalsize{}}

\newcommand{\Emph}[1]{\emph{\textcolor{blue}{#1}}}

\newcommand{\Cay}[1]{\operatorname{Cay}\left(#1\right)}
\newcommand{\Clique}[1]{\omega\left(#1\right)}
\newcommand{\diag}[1]{\operatorname{diag}\left(#1\right)}
\newcommand{\dual}[1]{\widetilde{#1}}
\newcommand{\support}[1]{\operatorname{supp}\left(#1\right)}
\newcommand{\weight}[1]{\operatorname{wt}\left(#1\right)}
\newcommand{\weightclass}[1]{\operatorname{wc}\left(#1\right)}

\newtheorem*{lemma}{Lemma}
\newtheorem{Lemma}{Lemma}
\newtheorem*{theorem}{Theorem}
\newtheorem{Theorem}{Theorem}
\newtheorem*{conjecture}{Conjecture}
\newtheorem{Conjecture}{Conjecture}
\newtheorem*{corollary}{Corollary}
\newtheorem{Corollary}[Lemma]{Corollary}
\newtheorem*{remark}{Remark}
\newtheorem{Remark}{Remark}
\newtheorem*{definition}{Definition}
\newtheorem{Definition}{Definition}
\newtheorem{Question}{Question}
\newtheorem{Questions}[Question]{Questions}
%
\newenvironment{proofof}[1]{\noindent\emph{Proof of #1.}}{\qed}

\title{Classifying bent functions by their Cayley graphs}

\author{
Paul~C.~Leopardi
\thanks{Australian Government -- Bureau of Meteorology
\protect\url{mailto:paul.leopardi@gmail.com}}
%\protect\texttt{mailto:paul.\-leopardi@anu.edu.au}}
}

\date{DRAFT: 19 November 2016}

\begin{document}

\maketitle

\begin{abstract}
%
%
\end{abstract}

% \section{Introduction}
% \label{sec-Introduction}
% Two recent papers \cite{Leo14Constructions,Leo15Twin} describe and investigate two infinite sequences of bent functions and their Cayley graphs.
% The bent function $\sigma_m$ on $\Z_2^{2 m}$ is described in the first paper \cite{Leo14Constructions}, on 
% generalizations of Williamson's construction for Hada\-mard matrices.
% The bent function $\tau_m$ on $\Z_2^{2 m}$ is described in the second paper \cite{Leo15Twin},
% which investigates some of the properties of the two sequences of bent functions.
% In this second paper it is shown that the bent functions $\sigma_m$ and $\tau_m$ both correspond to Hada\-mard difference sets with the same parameters
% \begin{align*}
% (v_m,k_m,\lambda_m,n_m) &= (4^m, 2^{2 m - 1} - 2^{m-1}, 2^{2 m - 2} - 2^{m-1}, 2^{2 m - 2}),
% \end{align*}
% and that their corresponding Cayley graphs are both strongly regular with the same parameters $(v_m,k_m,\lambda_m,\lambda_m)$.
% 
% The main result of the current paper is the following.
% \begin{Theorem}\label{HR-non-imomorphic-theorem}
% The Cayley graphs of the bent functions $\sigma_m$ and $\tau_m$ are isomorphic only when $m=1, 2,$ or $3.$ 
% \end{Theorem}
% 
% The remainder of the paper is organized as follows.
% Section \ref{sec-Background} outlines some of the background of this investigation.
% Section \ref{sec-Preliminaries} repeats some of the definitions and properties used in the previous papers \cite{Leo14Constructions, Leo15Twin},
% and includes further definitions used in the subsequent sections.
% Section~\ref{sec-Results} proves the main result, and resolves the conjectures and the question raised by the previous papers.
% Section~\ref{sec-Discussion} puts these results in context, and suggests future research.
% 
% \section{Background}\label{sec-Background}
% A recent paper of the author \cite{Leo14Constructions} describes a generalization of
% Williamson's construction for Hada\-mard matrices \cite{Wil44}
% using the real monomial representation of the basis elements of the Clifford algebras $\R_{m,m}$.
% In that paper, the following three conjectures appear:
% 
% \begin{Conjecture}\label{conjecture-1}
% %
% For all $m \geqslant 0$ there is a permutation $\pi$ of the set of $4^m$ canonical basis matrices,
% that sends an amicable pair of basis matrices with disjoint support to an anti-amicable pair, and vice-versa.
% %
% \end{Conjecture}
% 
% \begin{Conjecture}\label{conjecture-2}
% %
% For all $m \geqslant 0,$ 
% for the Clifford algebra $\R_{m,m},$ the subset of transversal graphs that are 
% not self-edge-colour complementary
% can be arranged into a set of pairs of graphs with each member of the pair 
% being edge-colour complementary to the other member.
% %
% \end{Conjecture}
% 
% \begin{Conjecture}\label{conjecture-3}
% %
% For all $m \geqslant 0,$ 
% for the Clifford algebra $\R_{m,m},$ if a graph $T$ exists amongst the transversal graphs,
% then so does at least one graph with edge colours complementary to those of $T$.
% %
% \end{Conjecture}
% 
% The author's subsequent paper on bent functions \cite{Leo15Twin} 
% refines Conjecture~\ref{conjecture-1} into the following question.
% \begin{Question}
% \label{Question-1}
% Consider the sequence of edge-coloured graphs $\varDelta_m$ for $m \geqslant 1$,
% each with red subgraph $\varDelta_m[-1],$ and blue subgraph $\varDelta_m[1].$
% For which $m \geqslant 1$ is there an automorphism of $\varDelta_m$ 
% that swaps the subgraphs $\varDelta_m[-1]$ and $\varDelta_m[1]$?
% \end{Question}
%  
% (The term \emph{transversal graph} and the definitions of $\varDelta_m$, $\varDelta_m[-1],$ and $\varDelta_m[1]$ 
% are given in the relevant papers and are repeated in the next section.)
% 
% The main result of this paper, Theorem \ref{HR-non-imomorphic-theorem} leads to the resolution of these conjectures and this question.

% \begin{frame}
\subsection*{Overview}
%\begin{center}
\begin{itemize}
\item
Key concepts.

~

\item
Equivalence.

~

\item
Some results.

~

\item
Observations for small dimensions.

~

\item
Some questions.

~

\item
SageMathCloud worksheet.
\end{itemize}
 
%\end{center}
% \end{frame}

\section{Key concepts}

% \begin{frame}
\subsection*{The Cayley graph of a binary function}
%\begin{center}
The \Emph{Cayley graph} $\Cay{f}$ of a binary function 

~

\begin{align*}
%
f : \Z_2^n \To \Z_2 \quad \text{where} \quad f(0) = 0 
% 
\end{align*}

~

is 
an undirected graph with 

\begin{align*}
V(\Cay{f}) &:= \Z_2^n, \quad (x,y) \in E(\Cay{f}) \Leftrightarrow f(x+y) = 1.
\end{align*}

~

\slidecite{Bernasconi and Codenotti 1999} % BerC99
% \end{frame}
% \begin{frame}
\subsection*{Bent functions}

A binary function  $f : \Z_2^{2m} \To \Z_2$ is \Emph{bent} if and only if the function $\dual{f}$, defined by
\begin{align*}
(-1)^{\dual{f}(x)} &:= 2^{-m} \sum_{y \in \Z_2^{2m}} (-1)^{f(y) + \langle x, y \rangle}
\end{align*}
is a binary function on $\Z_2^{2m}$.

~

The function $\dual{f}$ is also bent and is called the \Emph{dual} of $f$.

~

\slidecite{Dillon 1974; Rothaus 1976; Tokareva 2011}
% \end{frame}

% \begin{frame}
\subsection*{Strongly regular graphs}
%\begin{center}
A simple graph $\Gamma$ of order $v$ is \Emph{strongly regular} with parameters 
$(v,k,\lambda,\mu)$ if 

~

\begin{itemize}
 \item 
each vertex has degree $k,$ 

~
 \item 
each adjacent pair of vertices has $\lambda$ common neighbours, and

~
\item
each nonadjacent pair of vertices has $\mu$ common neighbours.
\end{itemize}

~

\slidecite{Brouwer, Cohen and Neumaier 1989} % BroCN89

%\end{center}
% \end{frame}

% \begin{frame}
\subsection*{Bent functions and strongly regular graphs}

\begin{Theorem}
\smallcite{Bernasconi and Codenotti 1999}

The Cayley graph $\Cay{f}$ of a bent function $f$ on $\Z_2^{2m}$ 

(with $f(0)=0$) is a strongly regular graph with $\lambda = \mu.$ 
\end{Theorem}

The parameters of $\Cay{f}$ are
\begin{align*}
(v,k,\lambda) = &(4^m, 2^{2 m - 1} - 2^{m-1}, 2^{2 m - 2} - 2^{m-1}) 
\\
  \text{or} \quad &(4^m, 2^{2 m - 1} + 2^{m-1}, 2^{2 m - 2} + 2^{m-1}).
\end{align*}

~

\slidecite{Menon 1962; Dillon 1974; Bernasconi and Codenotti 1999}
%\end{center}
% \end{frame}
% \begin{frame}
\subsection*{Weights and weight classes}
\begin{Definition}
The \Emph{weight} of a binary function is the cardinality of its \Emph{support}.
For $f$ on $\Z_2^{2m}$
\begin{align*}
\support{f} &:= \{x \in \Z_2^{2m} \mid f(x)=1 \}.  
\end{align*}

A bent function $f$ on $\Z_2^{2m}$ has weight
\begin{align*}
\weight{f} &= 2^{2 m - 1} - 2^{m-1} \quad (\text{weight class~} \weightclass{f}=0), \text{~or}
\\
\weight{f} &= 2^{2 m - 1} + 2^{m-1} \quad (\text{weight class~} \weightclass{f}=1).
\end{align*}
If $f(0)=0$ then $\weightclass{\Cay{f}} := \weightclass{f}$.
\end{Definition}
% \end{frame}

% \begin{frame}
\subsection*{The two block designs of a bent function}

The adjacency matrix of $\Cay{f}$ can also be interpreted as the incidence matrix of a block design.

~

In this case we do not need $f(0)=0$.

~
\begin{Definition}
A second block design described by Dillon and Schatz can be defined by the incidence matrix $D(f)$ where
\begin{align*}
D(f)_{c,x} &:= f(x) + \langle c, x \rangle + \dual{f}(c).   
\end{align*}
This is a symmetric block design with the \Emph{symmetric difference property}.
\end{Definition}

~

\slidecite{Dillon and Schatz 1987; Neumann 2006}
% \end{frame}
% \begin{frame}
\subsection*{Projective two-weight binary codes}

\begin{Definition}
A \Emph{two-weight binary code} with parameters $[n,k,d]$ is a $k$ dimensional subspace of $\Z_2^n$ with 
minimum Hamming distance $d$, such that the set of Hamming weights of the non-zero vectors has size 2.

~

``A \Emph{generator matrix} $G$ of a linear code $[n, k]$ code $C$ is any matrix
of rank $k$ (over $\Z_2$) with rows from $C.$''

~

``A linear $[n, k]$ code is called \Emph{projective} if no two columns of a generator matrix
$G$ are linearly dependent, i.e., if the columns of $G$ are pairwise different points in a
projective $(k-1)$-dimensional space.''
In the case of $\Z_2$, no two columns are equal.

~

\slidecite{Bouyukliev, Fack, Willems and Winne 2006} % BouFFWW2006

\end{Definition}

% \end{frame}
\section{Equivalence}

% \begin{frame}
\subsection*{Extended translation equivalence}

\begin{Definition}
For bent functions $f,g : \Z_2^{2m} \To \Z_2$, 

$f$ is \Emph{extended translation equivalent} to $g$ if and only if
\begin{align*}
g(x) &= f(x + b) + \langle c, x \rangle + \delta 
\end{align*}
for $b, c \in \Z_2^{2m}$, $\delta \in \Z_2$.
\end{Definition}
% \end{frame}

% \begin{frame}
\subsection*{Extended affine equivalence}

\begin{Definition}
For bent functions $f,g : \Z_2^{2m} \To \Z_2$, 

$f$ is \Emph{extended affine equivalent} to $g$ if and only if
\begin{align*}
g(x) &= f(A x + b) + \langle c, x \rangle + \delta 
\end{align*}
for some $A \in GL(2m,2)$, $b, c \in \Z_2^{2m}$, $\delta \in \Z_2$.
\end{Definition}
~

\slidecite{Tokareva 2014}
% \end{frame}
% \begin{frame}
\subsection*{Cayley equivalence}
\begin{Definition}
%
For $f, g : \Z_2^{2m} \To \Z_2$, with both $f$ and $g$ bent, 

we call $f$ and $g$ \Emph{Cayley equivalent},
and write $f \equiv g$, 

if and only if $f(0)=g(0)=0$ and $\Cay{f} \equiv \Cay{g}$ as graphs.

~

Equivalently, $f \equiv g$ if and only if $f(0)=g(0)=0$ and 

there exists a bijection $\pi : \Z_2^{2m} \To \Z_2^{2m}$ such that
\begin{align*}
g(x+y) &= f \big(\pi(x)+\pi(y)\big) \quad \text{for all~} x,y \in \Z_2^{2m}. 
\end{align*}
\end{Definition}
% \end{frame}
% \begin{frame}
\subsection*{Extended Cayley equivalence}
\begin{Definition}
For $f, g : \Z_2^{2m} \To \Z_2$, with both $f$ and $g$ bent,

if there exist $\delta, \epsilon \in \{0,1\}$ such that $f + \delta \equiv g + \epsilon$, 

we call $f$ and $g$ \Emph{extended Cayley (EC) equivalent} and write $f \cong g$. 
\end{Definition}
Extended Cayley equivalence is an equivalence relation on the set of all bent functions on $\Z_2^{2m}$.
% \end{frame}
\section{Some results}
% \begin{frame}

% \begin{frame}
\subsection*{Linear equivalence implies Cayley equivalence}

\begin{Theorem}
\label{th-Linear-Cayley}
If $f$ is bent with $f(0)=0$ and $g(x) := f(A x)$ where $A \in GL(2m,2)$,
then $g$ is bent with $g(0)=0$ and $f \equiv g$.
\end{Theorem}
\begin{proof}
\begin{align*}
g(x+y) &= f\big(A(x+y)\big) = f(A x + A y)\quad \text{for all~} x,y \in \Z_2^{2m}. 
\end{align*}
\end{proof}
 
% \end{frame}

% \begin{frame}
\subsection*{Extended affine, translation and Cayley equivalence}

\begin{Theorem}
\label{th-Affine-Translate-Cayley}
For $A \in GL(2m,2)$, $b, c \in \Z_2^{2m}$, $\delta \in \Z_2$,
$f : \Z_2^{2m} \To \Z_2$, 

the function
\begin{align*}
h(x) &:= f(A x + b) + \langle c, x \rangle + \delta
\intertext{can be expressed as $h(x) = g(A x)$ where}
g(x) &:= f(x+b) + \langle (A^{-1})^T c, x \rangle + \delta,
\end{align*}
and therefore if $f$ is bent and $h(0)=0$ then $h \equiv g$.
\end{Theorem}
% \end{frame}
\begin{proof}
Let $y:= A x$. Then
\begin{align*}
g(A x) = g(y) &= f(y+b) + \langle (A^{-1})^T c, y \rangle + \delta
\\
&= f(y+b) + \langle c, A^{-1} y \rangle + \delta
\\
&= f(A x + b) + \langle c, x \rangle + \delta = h(x).
\end{align*}
If $f$ is bent, then so are $g$ and $h$.
Therefore, by Theorem \ref{th-Linear-Cayley},
if $h(0)=0$ then $h \equiv g$.
\end{proof}


% \begin{frame}
% \subsection*{Extended affine equivalence}

Therefore, to determine the extended Cayley equivalence classes within the extended affine equivalence class of
a bent function $f : \Z_2^{2m} \To \Z_2$, for which $f(0)=0$, we need only examine 
the extended translation equivalent functions of the form
\begin{align*}
f(x+b) + \langle c, x \rangle + f(b),
\end{align*}
for each $b, c \in \Z_2^{2m}$.
% \end{frame}
% \begin{frame}
\subsection*{Quadratic bent functions have 2 EC classes}
\begin{Theorem}
\label{th-Quadratic-Classes}
For each $m>0$, the extended affine equivalence class of quadratic bent functions
$q : \Z_2^{2m} \To \Z_2$ contains exactly two extended Cayley equivalence classes,
corresponding to the two possible weight classes of $x \mapsto q(x+b) + \langle c, x \rangle + q(b)$. 
\end{Theorem}

The proof relies on a number of supporting lemmas.

\begin{Lemma}
\label{lm-notes-3}
Let $Z \in \Z_2^{2 m \times 2 m}$ be symmetric with zero diagonal.
In other words, $Z = Z^T$, $\diag{Z} = 0$.
Then for any $M \in \Z_2^{2 m \times 2 m}$,
\begin{align*}
x^T (M + Z) x  &= x^T M x
\end{align*}
for all $x \in \Z^{2 m}$.
\end{Lemma}

\begin{proof}
Let $Z$, $x$ be as above.
Then
\begin{align*}
x^T Z x
&=
\sum_{i=0}^{2m-1} \sum_{j=0}^{2m-1} x_i Z_{i,j} x_j
\\
&=
\sum_{i=0}^{2m-1} \sum_{j<i} x_i Z_{i,j} x_j +
\sum_{i=0}^{2m-1} x_i Z_{i,i} x_i +
\sum_{i=0}^{2m-1} \sum_{j>i} x_i Z_{i,j} x_j
\\
&=
\sum_{i=0}^{2m-1} \sum_{j<i} x_i (Z_{i,j} + Z_{j,i})
= 0.
\intertext{Therefore}
x^T (M + Z) x  &= x^T M x + x^T Z x = x^T M x.
\end{align*}
\end{proof}

\begin{Lemma}
\label{lm-notes-4}
Let $q(x) := x^T L x$ where $L \in \Z_2^{2 m \times 2 m}$,
\begin{align*}
L 
&:= 
\left[
\begin{array}{cc}
0 & I
\\
0 & 0
\end{array}
\right],
\intertext{so that}
q(x) &= \sum_{k=0}^{m-1} x_k x_{m+k}.
\end{align*}
Then for all $c \in Z_2^{2 m}$ with $q(c)=0$, there exists $A \in GL(2 m, 2)$ such that
\begin{align*}
q(A x) &= q(x) + \langle c, x \rangle.
\end{align*}
\end{Lemma}

\begin{proof}
Let $C \in \Z_2^{2 m \times 2 m}$ be such that $C_{i,j} = \delta_{i,j} c_i$, where $\delta$ is the
\Emph{Dirac delta}: $\delta_{i,j}=1$ if $i=j$ and $0$ otherwise. 
In other words $\diag{C} = c$.
Then
\begin{align*}
\langle c, x \rangle 
&= 
\sum_{i=0}^{2m-1} c_i x_i
\\
&= 
\sum_{i=0}^{2m-1} x_i c_i x_i
= 
x^T C x.
\end{align*}
Therefore, by Lemma \ref{lm-notes-3},
\begin{align*}
q(x) + \langle c, x \rangle
&=
x^T (L + Z + C) x,
\end{align*}
where $Z \in \Z_2^{2 m \times 2 m}$ is symmetric with zero diagonal.

For such $Z$, let $S := Z + C$. 
We want to find $A \in \Z_2^{2 m \times 2 m}$ such that $q(A x) = q(x) + \langle c, x \rangle.$
In other words,
\begin{align*}
q(A x) 
&= 
(A x)^T L (A x) 
= 
x^T A^T L A x 
= 
x^T (L + S) x.
\end{align*}
This will be true if $A^T L A = L + S.$

Let
\begin{align*}
A 
&:= 
\left[
\begin{array}{cc}
A_{0,0} & A_{0,1}
\\
A_{1,0} & A_{1,1}
\end{array}
\right],
\quad
S 
&:= 
\left[
\begin{array}{cc}
S_{0,0} & S_{0,1}
\\
S_{0,1}^T & S_{1,1}
\end{array}
\right]
=:
\left[
\begin{array}{cc}
Z_{0,0} + C_{0,0} & Z_{0,1}
\\
Z_{0,1}^T & Z_{1,1} + C_{1,1}
\end{array}
\right].
\end{align*}
Since
\begin{align*}
L A 
&=
\left[
\begin{array}{cc}
0 & I
\\
0 & 0
\end{array}
\right]
\left[
\begin{array}{cc}
A_{0,0} & A_{0,1}
\\
A_{1,0} & A_{1,1}
\end{array}
\right]
=
\left[
\begin{array}{cc}
A_{1,0} & A_{1,1}
\\
0 & 0
\end{array}
\right],
\end{align*}
we require that
\begin{align*}
A^T L A
&=
\left[
\begin{array}{cc}
A_{0,0} & A_{1,0}
\\
A_{0,1} & A_{1,1}
\end{array}
\right]
\left[
\begin{array}{cc}
A_{1,0} & A_{1,1}
\\
0 & 0
\end{array}
\right]
\\
&=
\left[
\begin{array}{cc}
A_{0,0}^T A_{1,0} & A_{0,0}^T A_{1,1}
\\
A_{0,1}^T A_{1,0} & A_{0,1}^T A_{1,1}
\end{array}
\right]
\\
&= 
L + S
=
\left[
\begin{array}{cc}
S_{0,0} & I + S_{0,1}
\\
S_{0,1}^T & S_{1,1}
\end{array}
\right],
\end{align*}
and therefore
\begin{align*}
A_{0,0}^T A_{1,0}
&=
S_{0,0},
\quad
A_{0,0}^T A_{1,1}
=
I + S_{0,1},
\\
A_{0,1}^T A_{1,0}
&=
S_{0,1}^T,
\quad
A_{0,1}^T A_{1,1}
=
S_{1,1}.
\end{align*}
If $S_{0,1}=0$ and $A_{0,0}=I$ then
$A_{1,0}=S_{0,0}$, $A_{1,1}=I$ and $A_{0,1}=S_{1,1}$.
In this case, we have $A_{0,1}^T A_{1,0} = S_{0,1}^T = 0$,
i.e. $S_{1,1} S_{0,0} = 0$, and 
\begin{align*}
A 
&= 
\left[
\begin{array}{cc}
I & S_{1,1}
\\
S_{0,0} & I
\end{array}
\right],
\intertext{so that}
A^T L A
&=
\left[
\begin{array}{cc}
I & S_{0,0}
\\
S_{1,1} & I
\end{array}
\right]
\left[
\begin{array}{cc}
S_{0,0} & I
\\
0 & 0
\end{array}
\right]
\\
&=
\left[
\begin{array}{cc}
S_{0,0} & I
\\
0 & S_{1,1}
\end{array}
\right]
\\
&= 
L + S.
\end{align*}

Also
\begin{align*}
S 
&= 
\left[
\begin{array}{cc}
Z_{0,0} + C_{0,0} & 0
\\
0 & Z_{1,1} + C_{1,1}
\end{array}
\right].
\end{align*}

Since $q(c)=0$ we have
\begin{align*}
q(c)
&=
\sum_{k=0}^{m-1} c_k c_{m+k}
=
0. 
\end{align*}
Let $K := \{ k \mid c_k c_{m+k} = 1 \}$.
Then we must have $\abs{K} = 2 r$ for some integer $r \geqslant 0$, i.e. $\abs{K}$ is even.
We therefore arbitrarily group the elements of $K$ into pairs $(i_p, j_p)$ for $p=0,\ldots,r-1$,
and define the matrix $T \in \Z_2^{m \times m}$ by
\begin{align*}
T_{i,j}
&:=
\sum_{p=0}^{r-1} (\delta_{i,i_p} \delta_{j,j_p} + \delta_{i,j_p} \delta_{j,i_p}),
\end{align*}
so that
\begin{align*}
\begin{cases}
T_{i_p,j_p}
=
T_{j_p,i_p}
=
1
&\text{for~} p \in \{0,\ldots,r-1\},
\\
T_{i,j} = 0
&\text{otherwise.} 
\end{cases}
\end{align*}
Since the $r$ pairs $(i_p, j_p)$ partition the set $K$, 
the matrix $T$ has at most one non-zero in each row and column.

Recalling that
\begin{align*}
(T^2)_{i,j}
&=
\sum_{k=0}^{m-1} T_{i,k} T_{k,j},
\end{align*}
we see that the general term $T_{i,k} T_{k,j}$ of this sum is non-zero only if either
\begin{align*}
\begin{cases}
i = j = i_p,&\text{and}\ k=j_p,\ \text{or}
\\
i = j = j_p,&\text{and}\ k=i_p,
\end{cases}
\end{align*}
for some $p \in \{0,\ldots,r-1\}$, with all $2r$ of these cases being mutually exclusive.
So $T^2$ is diagonal with $2r$ non-zeros at the elements of $K$.

But $C_{1,1} C_{0,0}$ is diagonal, and $(C_{1,1} C_{0,0})_{i,i} = c_{m+i} c_i$.
Therefore 
\begin{align}
T^2 &= C_{1,1} C_{0,0}.
\label{eq-t-2}
\end{align}

Now, let $Z_{0,0}=Z_{1,1}=T$. Then $S_{0,0} = T + C_{0,0}$, $S_{1,1} = T + C_{1,1}$, and
\begin{align*}
S_{1,1} S_{0,0} 
&= 
(T + C_{1,1})(T + C_{0,0}) 
= 
T^2 + T C_{0,0} + C_{1,1} T + C_{1,1} C_{0,0}
\\
&=
T C_{0,0} + C_{1,1} T,
\end{align*}
where in the last step, we have used \eqref{eq-t-2}.

Now,
\begin{align*}
(T C_{0,0} + C_{1,1} T)_{i,j} 
&= 
\sum_{k=0}^{m-1} T_{i,k} (C_{0,0})_{k,j} + (C_{1,1})_{i,k} T_{k,j}
\\
&=
T_{i,j} (C_{0,0})_{j,j} + (C_{1,1})_{i,i} T_{i,j}
\\
&= 
T_{i,j} \left( c_j + c_{m+i} \right).
\end{align*}
As above, $T_{i,j}$ is non-zero only when $(i,j)=(i_p,j_p)$ or $(i,j)=(j_p,i_p)$
for some $p \in \{0,\ldots,r-1\}$, but in all those cases $c_j=c_{m+j}=1$.

Therefore
\begin{align*}
S_{1,1} S_{0,0} &= T C_{0,0} + C_{1,1} T = 0.
\end{align*}
Similarly, $S_{0,0} S_{1,1} = 0$, and therefore
\begin{align*}
A^2
&=
\left[
\begin{array}{cc}
I & S_{1,1}
\\
S_{0,0} & I
\end{array}
\right]
\left[
\begin{array}{cc}
I & S_{1,1}
\\
S_{0,0} & I
\end{array}
\right]
\\
&=
\left[
\begin{array}{cc}
I + S_{1,1} S_{0,0} & S_{1,1} + S_{1,1}
\\
S_{0,0} + S_{0,0} & I + S_{0,0} S_{1,1}
\end{array}
\right]
=
\left[
\begin{array}{cc}
I & 0
\\
0 & I
\end{array}
\right].
\end{align*}


We have therefore shown that
\begin{align}
A 
&:= 
\left[
\begin{array}{cc}
I & T + C_{1,1}
\\
T + C_{0,0} & I
\end{array}
\right],
\quad
S 
:= 
\left[
\begin{array}{cc}
T + C_{0,0} & 0
\\
0 & T + C_{1,1}
\end{array}
\right]
\label{eq-a-s-def}
\end{align}
is a solution to $A^T L A = L + S$.

Finally, given $c$ with $q(c)=0$, the matrix $A$ as defined by \eqref{eq-a-s-def} is such that
$q(A x) = q(x) + \langle c, x \rangle$.
\end{proof}
% \end{frame}
\begin{Lemma}
\label{lm-notes-6}
For $k \in \{0,\ldots,m-1\}$ define $e^{(k)}$ by
\begin{align}
e_i^{(k)} &:= \delta_{i,k} + \delta_{i,m+k}
\label{eq-e-def} 
\end{align}
for $i \in \{0,\ldots,2 m - 1\}$.

Let $h(x) := q(x) + \langle e^{(0)}, x \rangle$, where $q$ is defined as per Lemma \ref{lm-notes-4}.
Then for any $c'$ such that $q(c')=1$, there exists $B \in \Z_2^{2 m \times 2 m}$ such that
\begin{align}
h(B x) &= q(x) + \langle c',x \rangle.
\label{eq-h-B-x}
\end{align}
TBD: should be $GL(2 m, 2)$.
\end{Lemma}

\begin{proof}
Let $K'=\{k \mid c'_k c'_{m+k} = 1\}$. Since $q(c')=1$, $\abs{K'}$ is odd.
Choose any $\ell \in K'$, and let $c := c' + e^{(\ell)}$.
Then $c_{\ell} = c_{m+\ell}$ and $q(c)=0$.

\end{proof}


% \begin{frame}
\subsection*{Dual functions}
\begin{Theorem}
\smallcite{Carlet 2007, Proposition 4} 

~

For a bent function $f : \Z_2^{2m} \To \Z_2$, and $b,c \in \Z_2^{2m}$,
if
\begin{align*}
g(x) &:= f(x+b) + \langle c, x \rangle
\intertext{then}
\dual{g}(x) &= \dual{f}(x+c) + \langle b, x \rangle + \langle b, c \rangle. 
\end{align*}
\end{Theorem}
 
% \end{frame}
% \begin{frame}
%\subsection*{Dual functions}
\begin{Theorem}
For a bent function $f : \Z_2^{2m} \To \Z_2$, and $A \in GL(2 m, 2)$,
if
\begin{align*}
g(x) &:= f(A x)
\intertext{then}
\dual{g}(x) &= \dual{f}\big((A^T)^{-1} x \big),
\end{align*}
\end{Theorem}
and therefore $\dual{g} \equiv \dual{f}$.
If, in addition, $f=\dual{f}$ then $\dual{g} \equiv g$. 
 
% \end{frame}
% \begin{frame}
%\subsection*{Dual functions}

Functions of the form 
\begin{align*}
f(x) := \sum_{k=0}^{m-1} x_{2k} x_{2k+1}
\end{align*}
are self dual bent functions, $f=\dual{f}$.

~

There are many other self dual bent functions.

~

\slidecite{Carlet, Danielson, Parker and Sol\'e 2008; Feulner, Sok, Sol\'e and Wassermann 2011} % FeuSSW2013
% \end{frame}
% \begin{frame}
\subsection*{The Dillon-Schatz design matrix}
\begin{Theorem}
For every bent function $f$, the \Emph{weight class matrix} of $f$ 
equals the incidence matrix of the Dillon-Schatz design of $f$.

~

Specifically,
\begin{align*}
\weightclass{\Cay{x \mapsto f(x+b) + \langle c, x \rangle + f(b)}}
&=
f(b) + \langle c, b \rangle + \dual{f}(c)
\\
&=
D(f)_{c,b}.
\end{align*}
 
\end{Theorem}

% \end{frame}
% \begin{frame}
\subsection*{From bent function to linear code}
\begin{Definition}

\smallcite{Ding 2015, Corollary 10} 

For a bent function $f : \Z_2^{2m} \To \Z_2$, 
define the linear code $C(f)$ by the generator matrix
\begin{align*}
M C(f)_{x,y} &\in \Z_2^{2^{2m} \times \weight{f}},
\\
M C(f)_{x,y} &:= \langle x, \support{f}(y) \rangle,
\end{align*}
with $x$ in lexicographic order of $\Z_2^{2m}$ 
and $\support{f}(y)$ in lexicographic order of $\support{f}$.

The $4^m$ words of the code $C(f)$ are the rows of the generator matrix $M C(f)$.
\end{Definition}

\slidecite{Ding 2015, Corollary 10} 
 
% \end{frame}
% \begin{frame}
%\frametitle{From bent function to linear code (2)}
\begin{Theorem}
\smallcite{Ding 2015, Corollary 10} 

For a bent function $f : \Z_2^{2m} \To \Z_2$, the linear code $C(f)$
is a two-weight projective binary code.

~

The possible weights of non-zero code words are:
\begin{align*}
\begin{cases}
2^{2m-2}, 2^{2m-2} - 2^{m-1} & \text{if~} \weightclass{f}=0.
\\
2^{2m-2}, 2^{2m-2} + 2^{m-1} & \text{if~} \weightclass{f}=1.
\end{cases}
\end{align*}

\end{Theorem}

\slidecite{Ding 2015, Corollary 10} 
 
% \end{frame}
% \begin{frame}
\subsection*{From linear code to strongly regular graph}
\begin{Definition}

Given $f : \Z_2^{2m} \To \Z_2$, form the linear code $C(f)$.

The graph $R(f)$ is defined as:

Vertices of $R(f)$ are code words of $C(f)$.

For $v,w \in C(f)$, edge $(u,v) \in R(f)$ if and only if
\begin{align*}
\begin{cases}
\weight{u+v} = 2^{2m-2} - 2^{m-1} & (\weightclass{f}=0).
\\
\weight{u+v} = 2^{2m-2} + 2^{m-1} & (\weightclass{f}=1).
\end{cases}
\end{align*}
 
\end{Definition}
Since $C(f)$ is a two-weight binary projective code, 
$R(f)$ is a strongly regular graph.
 
\slidecite{Delsarte 1972, Theorem 2}
% \end{frame}
% \begin{frame}
\subsection*{The graph $R(f)$ is the Cayley graph of the dual}

\begin{Theorem}
For $f : \Z_2^{2m} \To \Z_2$, with $f(0)=0$,
\begin{align*}
R(f) \equiv 
\begin{cases}
\Cay{\dual{f}} & \text{if~} \weightclass{f}=0,
\\
\Cay{\dual{f}+1} & \text{if~} \weightclass{f}=1.
\end{cases}
\end{align*}
 
\end{Theorem}
% \end{frame}

\section{Observations}
% \begin{frame}
\subsection*{For $m=1$}

One extended affine class: $[f_{2,1}]$ 

where $f_{2,1}(x) := x_0 x_1$ is self dual.

~

Two extended Cayley classes:
\begin{align*}
\begin{array}{|cccl|}
\hline
\text{Class} &
\text{Parameters} & 
\text{2-rank} &
\text{Clique polynomial}
\\
\hline
1 &
(4, 1, 0, 0) & 4 & 
2t^{2} + 4t + 1
\\
2 &
K_4 & 4 & 
t^{4} + 4t^{3} + 6t^{2} + 4t + 1
\\
\hline
\end{array}
\end{align*}

% \end{frame}
% \begin{frame}
\subsection*{For $m=2$}

One extended affine class: $[f_{4,1}]$ where 

$f_{4,1}(x) := x_0 x_1 + x_2 x_3$ is self dual.

~

Two extended Cayley classes:
\begin{align*}
\begin{array}{|cccl|}
\hline
\text{Class} &
\text{Parameters} & 
\text{2-rank} &
\text{Clique polynomial}
\\
\hline
1 &
(16, 6, 2, 2) & 
6 &
8t^{4} + 32t^{3} + 48t^{2} + 16t + 1
\\
2 &
(16, 10, 6, 6) & 
6 &
\begin{array}{l}
16t^{5} + 120t^{4} + 160t^{3} + 
\\
80t^{2} + 16t + 1
\end{array}
\\
\hline
\end{array}
\end{align*}
% \end{frame}
% \begin{frame}
%\subsection*{For $m=2$: two-weight codes}

The Cayley graphs for classes 1 and 2 are isomorphic to those those obtained from the following two-weight projective
codes:

\begin{align*}
\begin{array}{|ccc|}
\hline
\text{Class} &
\text{Parameters} & \text{Generator matrix}
\\
\hline
1 &
[6, 4, 2] & 
\left[
\begin{array}{cccccc}
0 & 0 & 1 & 1 & 1 & 1
\\
1 & 0 & 0 & 1 & 1 & 1
\\
1 & 1 & 1 & 1 & 0 & 0
\\
0 & 1 & 1 & 1 & 1 & 0
\end{array}
\right]
\\
2 &
[5, 4, 2] & 
\left[
\begin{array}{ccccc}
1 & 1 & 0 & 0 & 0
\\
0 & 1 & 1 & 0 & 0
\\
0 & 0 & 0 & 1 & 1
\\
1 & 0 & 0 & 0 & 1
\end{array}
\right]
\\
\hline
\end{array}
\end{align*}

% \end{frame}
% \begin{frame}
\subsection*{For $m=3$: extended affine classes}

Four extended affine classes:

\begin{align*}
\def\arraystretch{1.2}
\begin{array}{|cl|}
\hline
\text{Class} &
\text{Representative}
\\
\hline
\,[f_{6,1}] & f_{6,1} := 
\begin{array}{l}
x_{0} x_{1} + x_{2} x_{3} + x_{4} x_{5}
\end{array}
\\
\,[f_{6,2}] & f_{6,2} := 
\begin{array}{l}
x_{0} x_{1} x_{2} + x_{0} x_{3} + x_{1} x_{4} + x_{2} x_{5}
\end{array}
\\
\,[f_{6,3}] & f_{6,3} := 
\begin{array}{l}
x_{0} x_{1} x_{2} + x_{0} x_{1} + x_{0} x_{3} + x_{1} x_{3} x_{4} + x_{1} x_{5} + 
\\
x_{2} x_{4} + x_{3} x_{4}
\end{array}
\\
\,[f_{6,4}] & f_{6,4} := 
\begin{array}{l}
x_{0} x_{1} x_{2} + x_{0} x_{3} + x_{1} x_{3} x_{4} + x_{1} x_{5} + x_{2} x_{3} x_{5} + 
\\
x_{2} x_{3} + x_{2} x_{4} + x_{2} x_{5} + x_{3} x_{4} + x_{3} x_{5}
\end{array}
\\
\hline
\end{array}
\end{align*}
% \end{frame}
% \begin{frame}
\subsection*{For EA class $[f_{6,1}]$}

The function
$f_{6,1}(x) = x_0 x_1 + x_2 x_3 + x_4 x_5$ is self dual.

~

Two extended Cayley classes:
\small{}
\begin{align*}
\def\arraystretch{1.2}
\begin{array}{|cccl|}
\hline
\text{Class} &
\text{Parameters} & 
\text{2-rank} &
\text{Clique polynomial}
\\
\hline
1 &
(64, 28, 12, 12) & 8 & 
\begin{array}{l}
64t^{8} + 512t^{7} + 1792t^{6} + 3584t^{5} +
\\
5376t^{4} + 3584t^{3} + 896t^{2} + 64t + 1
\end{array}
\\
2 &
(64, 36, 20, 20) & 8 & 
\begin{array}{l}
2304t^{6} + 13824t^{5} + 19200t^{4} + 
\\
7680t^{3} + 1152t^{2} + 64t + 1
\end{array}
\\
\hline
\end{array}
\end{align*}
% \end{frame}
% \begin{frame}
\subsection*{For EA class $[f_{6,1}]$}

The Cayley graphs for classes 1 and 2 are isomorphic to those those obtained from the following two-weight projective
codes as listed by Tonchev (2006):

\begin{align*}
\begin{array}{|ccl|}
\hline
\text{Class} &
\text{Parameters} & \text{Reference}
\\
\hline
1 & [35,6,16] & \text{Tonchev Table 1.56 1, 2 }
\\
2 & [27,6,12] & \text{Tonchev Table 1.55 1 }
\\
\hline
\end{array}
\end{align*}

\slidecite{Tonchev 1996, 2006}
% \end{frame}
% \begin{frame}
\subsection*{For EA class $[f_{6,2}]$}

The function
$f_{6,2}(x) = x_{0} x_{1} x_{2} + x_{0} x_{3} + x_{1} x_{4} + x_{2} x_{5}$.

~

Three extended Cayley classes:
\small{}
\begin{align*}
\def\arraystretch{1.2}
\begin{array}{|cccl|}
\hline
\text{Class} &
\text{Parameters} & 
\text{2-rank} &
\text{Clique polynomial}
\\
\hline
1 &
(64, 28, 12, 12) & 8 &
\begin{array}{l}
64t^{8} + 512t^{7} + 1792t^{6} + 3584t^{5} + 
\\
5376t^{4} + 3584t^{3} + 896t^{2} + 64t + 1
\end{array}
\\
2 &
(64, 28, 12, 12) & 8 & 
\begin{array}{l}
256t^{6} + 1536t^{5} + 4352t^{4} + 
\\
3584t^{3} + 896t^{2} + 64t + 1
\end{array}
\\
3 &
(64, 36, 20, 20) & 8 &
\begin{array}{l}
192t^{8} + 1536t^{7} + 8960t^{6} + 19968t^{5} +
\\
20224t^{4} + 7680t^{3} + 1152t^{2} + 64t + 1
\end{array}
\\
\hline
\end{array}
\end{align*}
Graph 1 is isomorphic to graph 1 of EA class $[f_{6,1}]$, and is also isomorphic to the complement of Royle's $(64,35,18,20)$ SRG $X$.

~

\slidecite{Royle 2008}
% \end{frame}
% \begin{frame}
%\subsection*{For EA class $[f_{6,2}]$: two-weight codes}

The Cayley graphs for classes 1 to 3 are isomorphic to those those obtained from the following two-weight projective
codes as listed by Tonchev (2006):

\begin{align*}
\begin{array}{|ccl|}
\hline
\text{Class} &
\text{Parameters} & \text{Reference}
\\
\hline
1 & [35,6,16] & \text{Tonchev Table 1.56 1, 2 }
\\
2 & [35,6,16] & \text{Tonchev Table 1.56 3 }
\\
3 & [27,6,12] & \text{Tonchev Table 1.55 2 }
\\
\hline
\end{array}
\end{align*}

\slidecite{Tonchev 1996, 2006}
% \end{frame}
% \begin{frame}
\subsection*{For EA class $[f_{6,3}]$}

The function
\begin{align*}
f_{6,3}(x) &= x_{0} x_{1} x_{2} + x_{0} x_{1} + x_{0} x_{3} + x_{1} x_{3} x_{4} 
\\
           &+ x_{1} x_{5} + x_{2} x_{4} + x_{3} x_{4}.
\end{align*}

Four extended Cayley classes:
\small{}
\begin{align*}
\def\arraystretch{1.2}
\begin{array}{|cccl|}
\hline
\text{Class} &
\text{Parameters} & 
\text{2-rank} &
\text{Clique polynomial}
\\
\hline
1 &
(64, 28, 12, 12) & 12 & 
\begin{array}{l}
32t^{8} + 256t^{7} + 896t^{6} + 2048t^{5} +
\\
4608t^{4} + 3584t^{3} + 896t^{2} + 64t + 1
\end{array}
\\
2 &
(64, 28, 12, 12) & 12 & 
\begin{array}{l}
64t^{6} + 1024t^{5} + 4096t^{4} + 
\\
3584t^{3} + 896t^{2} + 64t + 1
\end{array}
\\
3 &
(64, 36, 20, 20) & 12 & 
\begin{array}{l}
160t^{8} + 1280t^{7} + 9344t^{6} + 21504t^{5} +
\\
20480t^{4} + 7680t^{3} + 1152t^{2} + 64t + 1
\end{array}
\\
4 &
(64, 36, 20, 20) & 12 & 
\begin{array}{l}
160t^{8} + 1664t^{7} + 9792t^{6} + 21504t^{5} +
\\
20480t^{4} + 7680t^{3} + 1152t^{2} + 64t + 1
\end{array}
\\
\hline
\end{array}
\end{align*}
% \end{frame}
% \begin{frame}
%\subsection*{For EA class $[f_{6,3}]$: two-weight codes}

The Cayley graphs for classes 1 to 4 are isomorphic to those those obtained from the following two-weight projective
codes as listed by Tonchev (2006):

\begin{align*}
\begin{array}{|ccl|}
\hline
\text{Class} &
\text{Parameters} & \text{Reference}
\\
\hline
1 & [35,6,16] & \text{Tonchev Table 1.56 4 }
\\
2 & [35,6,16] & \text{Tonchev Table 1.56 5 }
\\
3 & [27,6,12] & \text{Tonchev Table 1.55 3 }
\\
4 & [27,6,12] & \text{Tonchev Table 1.55 4 }
\\
\hline
\end{array}
\end{align*}

\slidecite{Tonchev 1996, 2006}
% \end{frame}
% \begin{frame}
\subsection*{For EA class $[f_{6,4}]$}

The function
\begin{align*}
f_{6,4}(x) &= x_{0} x_{1} x_{2} + x_{0} x_{3} + x_{1} x_{3} x_{4} + x_{1} x_{5} + x_{2} x_{3} x_{5} 
\\
           &+ x_{2} x_{3} + x_{2} x_{4} + x_{2} x_{5} + x_{3} x_{4} + x_{3} x_{5}.
\end{align*}

Three extended Cayley classes:
\small{}
\begin{align*}
\def\arraystretch{1.2}
\begin{array}{|cccl|}
\hline
\text{Class} &
\text{Parameters} & 
\text{2-rank} &
\text{Clique polynomial}
\\
\hline
1 &
(64, 28, 12, 12) & 14 &
\begin{array}{l}
32t^{8} + 256t^{7} + 896t^{6} + 1792t^{5} +
\\
4480t^{4} + 3584t^{3} + 896t^{2} + 64t + 1
\end{array}
\\
2 &
(64, 28, 12, 12) & 14 &
\begin{array}{l}
16t^{8} + 128t^{7} + 448t^{6} + 1280t^{5} +
\\
4224t^{4} + 3584t^{3} + 896t^{2} + 64t + 1
\end{array}
\\
3 &
(64, 36, 20, 20) & 14 &
\begin{array}{l}
176t^{8} + 1408t^{7} + 9664t^{6} + 22272t^{5} +
\\
20608t^{4} + 7680t^{3} + 1152t^{2} + 64t + 1
\end{array}
\\
\hline
\end{array}
\end{align*}
% \end{frame}
% \begin{frame}
%\subsection*{For EA class $[f_{6,4}]$: two-weight codes}

The Cayley graphs for classes 1 to 3 are isomorphic to those those obtained from the following two-weight projective
codes as listed by Tonchev (2006):

\begin{align*}
\begin{array}{|ccl|}
\hline
\text{Class} &
\text{Parameters} & \text{Reference}
\\
\hline
1 & [35,6,16] & \text{Tonchev Table 1.56 7 }
\\
2 & [35,6,16] & \text{Tonchev Table 1.56 6 }
\\
3 & [27,6,12] & \text{Tonchev Table 1.55 5 }
\\
\hline
\end{array}
\end{align*}

\slidecite{Tonchev 1996, 2006}
% \end{frame}
\section{Questions}
% \begin{frame}
%\subsection*{Questions (1)}
(Preceded by the values of $m$ for which the question is settled.)
 
\begin{description}
\item[1-3]
How many EC classes are there for each $m$? ``Exponential numbers'' of classes? 
\item[1-3]
Is the number of EC classes within an EA class bounded? What is the bound?
\item[1-4]
Is the number of EC classes within the quadratic EA class always 2?
\item[1-3]
Which EC classes overlap more than one EA class?
\end{description}

\slidecite{Kantor 1983; Jungnickel and Tonchev 1991}
% \end{frame}
% \begin{frame}
%\subsection*{Questions (2)} 

\begin{description}
\item[1-3]
Which bent functions are Cayley equivalent to their dual?
\item[1-3]
Which EC classes contain a self-dual bent function?
\item[2-3]
Which EC classes correspond to projective two-weight codes?
\item[1-3]
For each EA representative $f$, what is the relationship between the Dillon-Schatz SDP incidence matrix
\begin{align*}
D(f)_{c,x} &= f(x) + \langle c, x \rangle + \dual{f}(c)
\intertext{and the matrix of EC classes}
M_{c,b} &= \big[\Cay{x \mapsto f(x+b) + \langle c, x \rangle + f(b)}\big] ?
\end{align*}
\end{description}
% \end{frame}
\section{SageMathCloud}
% \begin{frame}[fragile]
%\subsection*{Public worksheet on SageMathCloud}
~

See
\begin{verbatim}
http://tinyurl.com/jnchhev
\end{verbatim}
 
% \end{frame}

%%%%%%%%%%%%%%%%%%%%%%%%%%%%%%%%%%%%%%%%%%%%%%%%%%%%%%%%%%%%%%
\paragraph*{Acknowledgements.}

Thanks to Christine Leopardi for her hospitality at Long Beach.
Thanks to Robert Craigen, Joanne Hall, William Martin,
Padraig {\'O} Cath{\'a}in and Judy-anne Osborn for valuable discussions.
This work was begun in 2014 while the author was a Visiting Fellow at the Australian National University, 
and concluded while the author was a Visiting Fellow and a Casual Academic at the University of Newcastle, Australia.
%Thanks also to the anonymous reviewer of the previous draft of this paper.

%%%%%%%%%%%%%%%%%%%%%%%%%%%%%%%%%%%%%%%%%%%%%%%%%%%%%%%%%%%%%%%%%%%%%%

%\addtocontents{toc}{\vspace{0.5cm}}

%\newpage

%\bibliographystyle{abbrv-par}

%\bibliography{bib}

%\input{AJC-2015-Leopardi-Twin-bent-Hurwitz-Radon-revised.bbl}

\end{document}
% ----------------------------------------------------------------
