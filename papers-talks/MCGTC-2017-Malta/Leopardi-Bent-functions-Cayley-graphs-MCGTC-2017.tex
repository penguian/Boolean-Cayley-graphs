%%%%%%%%%%%%%%%%%%%%%%%%%%%%%%%%%%%%%%%%%%%%%%%%%%%%%%%%
%%%%%%%% Please do not modify the preamble  %%%%%%%%%%%%
%%%%%%%%%%%%%%%%%%%%%%%%%%%%%%%%%%%%%%%%%%%%%%%%%%%%%%%%

\documentclass[a4paper,12pt]{article}
\usepackage[top=3.5cm,bottom=2.5cm,left=2.5cm,right=2.5cm]{geometry}

\usepackage{amsmath,amsthm,amsfonts,amssymb,latexsym,enumerate,graphicx,parskip}

\newcommand{\titleoftalk}[1]{{\LARGE \textbf{#1}}\\ [20pt]}
\newcommand{\speaker}[1]{{\large #1}\\ [10pt]}
\newcommand{\university}[1]{\textsc{#1}\\ [10pt]}
\newcommand{\email}[1]{{\texttt{\small #1}}\\ [20pt]}
\newcommand{\joint}[1]{{\small #1}\\ [40pt]}

\begin{document}

%%%%%%%%%%%%%%%%%%%%%%%%%%%%%%%%%%%%%%%%%%%%%
%%%% Fill in the required details below. %%%%


\begin{centering}
\titleoftalk{Classifying bent functions by their Cayley graphs}  %%% input title of talk
\speaker{{Paul} {Leopardi}}      %%% input first-name and surname of presenting author
\university{University of Melbourne, \\ Australian Government - Bureau of Meteorology}   %%% input affiliation of presenting author
\email{paul.leopardi@gmail.com}     %%% input e-mail address of presenting author
%\joint{(\textit{joint work with }{Name} {Coauthor1}, {Name} {Coauthor2})} %%%comment or delete this line if not relevant
\end{centering}

\

\begin{abstract}
Bent Boolean functions are fascinating and useful combinatorial objects, whose applications include
coding theory and cryptography. The number of bent functions explodes with dimension, and various
concepts of equivalence are used to classify them. In 1999 Bernasconi and Codenotti [1] noted that the
Cayley graph of a bent function is strongly regular. This talk describes the concept of extended
Cayley equivalence of bent functions, discusses some connections between bent functions, designs,
and codes, and explores the relationship between extended Cayley equivalence and extended affine
equivalence. SageMath scripts and SageMathCloud worksheets [2] are used to compute and display some
of these relationships, for bent functions up to dimension 8.

%% No references are required, but if you give any, please use the following format and arrange in ALPHABETICAL ORDER according to surname.

%% Delete the following if no references are given.
%%% *****************

\vspace{25pt}

 \setlength{\parindent}{0cm}{\textbf{References:}

% Journal paper
[1] A.~Bernasconi and B.~Codenotti, Spectral analysis of {Boolean} functions as a graph eigenvalue
  problem, \textit{IEEE Transactions on Computers} \textbf{48(3)} (1999) 345--351.
A.~Bernasconi and B.~Codenotti.

[2] SageMath, Inc. \textit{SageMathCloud Online Computational Mathematics}, (2016).
}
%%% *****************

\end{abstract}


\end{document}
